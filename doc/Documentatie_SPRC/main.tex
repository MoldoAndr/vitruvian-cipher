\documentclass[12pt,oneside]{report}

% Pachete de bază (alese din cadrul șablonului robust)
\usepackage[utf8]{inputenc}
\usepackage[a4paper, top=2cm, bottom=2cm, left=2.5cm, right=2.5cm]{geometry}
\usepackage[
    backend=bibtex,
    style=numeric,
    bibencoding=ascii,
    sorting=anyt
]{biblatex}
\usepackage{caption}
\usepackage{datetime}
\usepackage{etoolbox}
\usepackage{fancyhdr}
\usepackage{float}
\usepackage{graphics}
\usepackage{graphicx}
\usepackage{svg}
\usepackage{hyperref}
\usepackage{lastpage}
\usepackage{lscape}
\usepackage{subcaption}
\usepackage{tablefootnote}
\usepackage[titles]{tocloft}
\usepackage{titlesec}
\usepackage{xcolor}
\colorlet{gray}{black} % ensure any gray placeholders render as black

\hypersetup{hidelinks}
\graphicspath{{images/}}
\svgpath{{images/}}
\addbibresource{bibliography.bib}

% Formatare titluri și liste
\setlength{\cftfigindent}{0em}
\setlength{\cfttabindent}{0em}
\renewcommand*\contentsname{Cuprins}
\titleformat{\chapter}[display]{\bfseries\huge}{Capitolul \thechapter:}{0pt}{}[]

% Stiluri pentru antet/subsol
\setlength{\headheight}{14pt}
% Micșorează spațiul dintre antet (NECLASIFICAT) și text
\setlength{\headsep}{8pt}
\setlength{\footskip}{16pt}
\renewcommand{\headrulewidth}{1pt}
\renewcommand{\footrulewidth}{1pt}
\fancypagestyle{front}{
  \fancyhf{}
  \fancyhead[C]{\small NECLASIFICAT}
  \fancyfoot[C]{\small NECLASIFICAT\\\thepage}
}
\pagestyle{front}
\fancypagestyle{plain}{
  \fancyhf{}
  \fancyhead[C]{\small NECLASIFICAT}
  \fancyfoot[C]{\small NECLASIFICAT\\\thepage\ din \pageref*{LastPage}}
}
\newenvironment{frontmatter}{\pagenumbering{roman}}{\newpage \pagenumbering{arabic}}
\AtBeginDocument{\addtocontents{toc}{\protect\thispagestyle{front}}}
\addtocontents{lof}{\protect\thispagestyle{front}}

% Comenzi auxiliare
\newcommand{\textbfit}[1]{\textbf{\textit{#1}}}
\newenvironment{codeenv}{\fontfamily{cmtt}\selectfont}{\par}
\DeclareTextFontCommand{\inlinecodenonboxed}{\codeenv}
\newcommand{\inlinecode}[1]{\mbox{\inlinecodenonboxed{#1}}}
\newenvironment{abbreviations}{\begin{list}{}{\renewcommand{\makelabel}{\abbrlabel}}}{\end{list}}
\newcommand{\abbrlabel}[1]{\makebox[3cm][l]{\textbf{#1}\ \dotfill}}

% Configurare metadate (editabil rapid)
\newcommand{\detaildate}{\color{gray}2025}
\newcommand{\detailcountry}{România}
\newcommand{\detailcity}{București}
\newcommand{\detailparentinstitution}{Ministerul Apărării Naționale}
\newcommand{\detailuniversity}{Academia Tehnică Militară ``\textit{Ferdinand I}''}
\newcommand{\detailfaculty}{Facultatea de Sisteme Informatice și Securitate Cibernetică}
\newcommand{\detailspecialization}{\color{gray}Calculatoare și Sisteme Informatice pentru Securitate si Apărare Națională}
\newcommand{\detailtitle}{\color{gray}Unelte software bazate pe mecanisme de inteligență artificială aplicată în criptografie}
\newcommand{\detailauthor}{\color{gray}Std. Sg. Maj. Moldovan Andrei-Gabriel}
\newcommand{\detailadviser}{\color{gray}Cpt. Subașu Georgiana-Ramona}

\begin{document}

\begin{frontmatter}

% Pagina de titlu
\begin{titlepage}
    \begin{center}

        \textbf{\detailcountry} \\
        \textbf{\detailparentinstitution} \\
        \textbf{\detailuniversity} \\

        \vspace*{0.5cm}

        \textbf{\detailfaculty} \\
        \textbf{\detailspecialization} \\

        \vspace*{1.5cm}

        \includegraphics[width=0.4\textwidth]{ATM.png}

        \vspace*{1.5cm}

        \large{\textbf{\detailtitle}}

        \vspace*{0.5cm}

        \normalsize
        \begin{tabular*}{\textwidth}{l@{\extracolsep{\fill}}r}
            \textbf{Coordonator Științific} & \textbf{Absolvent}\\
            \footnotesize \detailadviser & \footnotesize \detailauthor
        \end{tabular*}

        \vspace*{3cm}

        \vspace*{1.5cm}
        \textbf{\detailcity} \\
        \textbf{\detaildate}

    \end{center}
\end{titlepage}

% Paginare pentru frontmatter
\setcounter{page}{1}

% Abstract (RO)
\chapter*{Abstract}
\thispagestyle{front}
{\color{gray}
Lucrarea propune o platformă integrată care combină diferiți agenți AI specializați ca utilitare de criptanaliză și audit de parole pentru a acoperi un flux complet de securitate modernă din punct de vedere criptografic. Sistemul folosește orchestrare prin microservicii containerizate, momentan, și include: un agent de decizie (SecureBERT) pentru extragere de intenție și de entități relevante; un detector de criptosisteme inspirat din CyberChef/dcode.fr; un orchestrator de spargere de hash‑uri cu HashCat și John și generare de parole folosind PassGAN; un agregator de scoruri de parole (rețea neuronală, zxcvbn și HaveIBeenPwned), încă în lucru; un serviciu de testare primalitate/factorizare (YAFU + FactorDB); și un RAG local pentru asistență teoretică în criptografie. Se urmărește realizarea unei platforme centralizate care poate fi folosită local sau în cloud, cu capabilități AI/ML pentru detecție, analiză și suport educațional, punând accent pe modularitate, extindere ulterioară către Kubernetes și aliniere la bune practici de securitate.
}

\bigskip

{\color{gray}
    \textbf{Cuvinte cheie:} inteligență artificială, criptanaliză, securitate cibernetică, parole, HashCat, John the Ripper, PassGAN, SecureBERT, RAG local, containerizare, microservicii, Kubernetes, FactorDB, YAFU, zxcvbn, HaveIBeenPwned.
}

\newpage

% Cuprins + liste
\setcounter{secnumdepth}{3}
\phantomsection
\tableofcontents
\newpage

% Lista de abrevieri
\chapter*{Listă de Abrevieri}
\thispagestyle{front}
\begin{abbreviations}
    \item[AEAD] Authenticated Encryption with Associated Data
    \item[AES] Advanced Encryption Standard
    \item[AI] Artificial Intelligence
    \item[API] Application Programming Interface
    \item[CI/CD] Continuous Integration / Continuous Deployment
    \item[CLI] Command Line Interface
    \item[CTF] Capture The Flag
    \item[DES/3DES] Data Encryption Standard / Triple DES
    \item[ECC] Elliptic Curve Cryptography
    \item[ECDH] Elliptic Curve Diffie--Hellman
    \item[ETSI] European Telecommunications Standards Institute
    \item[FIPS] Federal Information Processing Standards
    \item[HIBP] Have I Been Pwned
    \item[HMAC] Hash-based Message Authentication Code
    \item[LLM] Large Language Model
    \item[MAC] Message Authentication Code
    \item[ML] Machine Learning
    \item[NIST] National Institute of Standards and Technology
    \item[ONNX] Open Neural Network Exchange
    \item[PKI] Public Key Infrastructure
    \item[PQC] Post-Quantum Cryptography
    \item[RAG] Retrieval-Augmented Generation
    \item[REST] Representational State Transfer
    \item[RFC] Request for Comments
    \item[RSA] Rivest--Shamir--Adleman
    \item[TLS] Transport Layer Security
    \item[WAF] Web Application Firewall
\end{abbreviations}
\newpage

\end{frontmatter}

% -------------------------
% Conținut capitole
% -------------------------
\pagestyle{plain}

\chapter{Introducere}
\label{ch:introduction}

\section{Context general}
{\color{gray}
Criptografia modernă a evoluat cu pași repezi, de la algoritmi simetrici clasici (DES, 3DES), către standarde robuste precum AES și suitele asimetrice (RSA, ECC), odată cu dezvoltarea infrastructurilor PKI și a protocoalelor TLS, precum și apariției PQC. În paralel, progresul AI a adus atât beneficii considerabile, cât și riscuri: modelele de tip ML/LLM pot accelera detecția anomaliilor și analiza traficului criptat, putând fi folosite și pentru generarea de atacuri (phishing, parole, malware) sau pentru asistarea criptanalizei. Astfel, securitatea cibernetică integrează din ce în ce mai mult tehnici AI precum răspuns rapid, corelare de evenimente și întărirea mecanismelor criptografice.
}

\section{Problema abordată}
{\color{gray}
Problema principală urmărește modul în care putem folosi agenți AI specializați pentru a automatiza analiza și răspunsul la amenințări criptografice, de la identificarea algoritmilor și auditul parolelor, până la factorizare pentru testare de securitate, cât și modul în care se poate îmbunătăți educația criptografică. Se observă, de asemenea, lipsa unei platforme integrate care să ofere aceste capabilități într-un mod modular, scalabil și sigur.
}

\section{Scopul lucrării}
{\color{gray}
Scopul lucrării este reprezentat de proiectarea unei platforme integrate, modulare și scalabile care folosește agenți AI pentru identificare de algoritmi, audit de parole, factorizare / teste criptografie, asistență teoretică, astfel încât detecția, analiza și răspunsul la amenințări să fie automatizate, reproductibile și ușor de folosit, atât local, cât și în cloud.
}

\section{Obiective specifice}
\begin{itemize}
    \item Realizarea și testarea robustă a unei părți din agenții AI propuși.
    \item Integrarea tuturor agenților într-o infrastructură comună.
    \item Crearea unei soluții de deploy și testare.
    \item Crearea unei arhitecturi de rețea care să asigure comunicația securizată.
\end{itemize}

\chapter{Fundamente Teoretice}
\label{ch:foundations}

\section{Elemente de bază ale criptografiei}
Criptografia urmărește să asigure confidențialitate, integritate, autentificare și nerepudiere pentru date transmise pe canale nesigure. Se împart două familii majore: criptografia simetrică (aceeași cheie pentru criptare/decriptare) și criptografia asimetrică (pereche publică/privată), completate de funcții hash și coduri de autentificare a mesajelor (MAC/HMAC) \cite{stinson,schneier,stallings}. Un criptosistem se definește formal prin spațiul mesajelor, al textelor cifrate, al cheilor și printr-o familie de funcții de criptare/decriptare parametrizate de cheie, iar securitatea se descrie prin modele de adversar și noțiuni riguroase de rezistență la atac. În practică, protocoalele moderne combină aceste primitive cu management de chei (generare, stocare, rotație) și cu mecanisme de transport securizat (ex. TLS), fiind fundamentele pe care se sprijină soluția propusă.

\subsection{Definiții și clasificare}
Din punct de vedere al clasificării, criptografia acoperă o suită vastă de primitive: criptare simetrică, criptare asimetrică, diverse criptosisteme (bloc, flux, hibrid), mecanisme de autentificare, funcții hash, semnături digitale, management al cheilor, generatoare pseudo-aleatoare și protocoale de distribuire a secretelor. Criptografia simetrică utilizează aceeași cheie pentru criptare și decriptare, oferind performanțe foarte bune și fiind preferată pentru volume mari de date, dar presupune existența unui canal sigur pentru distribuirea cheilor (exemplu: AES). Criptografia asimetrică (cu cheie publică) folosește o pereche publică/privată pentru a facilita schimbul securizat de chei, autentificarea și semnăturile digitale (RSA, ElGamal), însă costul computațional este mai ridicat.

Funcțiile hash criptografice produc un digest de lungime fixă dintr-un mesaj de lungime variabilă, fiind concepute să reziste la inversare și coliziuni, motiv pentru care sunt folosite la verificarea integrității și la stocarea sigură a parolelor. Alte primitive importante includ codurile de autentificare a mesajului (MAC/HMAC), care combină o cheie secretă cu o funcție hash sau cu o schemă de criptare pentru a valida autenticitatea și integritatea; scheme de semnătură digitală ce oferă non-repudiere fără partajarea cheii; generatoare de numere aleatoare criptografic sigure; și protocoale de negociere a cheilor, precum Diffie–Hellman sau ECDH. În practica modernă, sistemele combină aceste primitive: asimetria este folosită pentru a stabili secrete, simetria pentru traficul intens, hash-ul și MAC-urile pentru integritate și autentificare, iar managementul cheilor (generare, rotație, revocare) asigură durabilitatea modelului de securitate \cite{handbook_applied_crypto}.

\subsection{Standardizare}
Organismele de standardizare în criptografie sunt esențiale pentru a asigura interoperabilitatea între sisteme, validarea algoritmilor criptografici și adaptarea tehnologiei la amenințările emergente, inclusiv cele cuantice. Exemple cheie includ NIST (FIPS 140 pentru module criptografice, FIPS 197/AES și seria FIPS 203–205 pentru algoritmi post-cuanți), ISO/IEC (ISO/IEC 19790 privind cerințele de securitate și ISO/IEC 18033 pentru algoritmi de criptare), ETSI (TS 104 015 dedicat criptografiei cuantice), ITU (familia ITU-T Y.3800 pentru distribuția cuantică de chei), BSI (TR-02102 cu recomandări de algoritmi) și IETF (RFC 5280 pentru PKI X.509). Ele stabilesc cerințele de certificare, mecanismele de testare și ghidurile de implementare care ghidează atât soluțiile civile, cât și cele militare \cite{cryptographic_standardization}.

\section{Elemente fundamentale de inteligență artificială}
Se folosesc mai multe concepte de IA aplicate:
\begin{itemize}
    \item învățare supervizată și clasificare semantică;
    \item embeddings obținute cu arhitecturi Transformer;
    \item RAG, care combină recuperarea de fragmente, reranking;
    \item reranking al rezultatelor de căutare pe baza scorurilor ML;
    \item data augmentation în generatorul de întrebări (questions\_generator) pentru a extinde seturile de antrenare cu variații și parafrazări sintetice;
    \item ensemble și agregare de scoruri (password checker, rețea neuronală, zxcvbn și HIBP);
    \item inferență optimizată prin ONNX pentru performanță și portabilitate;
\end{itemize}

\section{Conexiunea AI--Criptografie}
AI-ul este din ce în ce mai implicat în criptografie, atât în analiză, cât și ca obiect al securizării. Modele AI specializate pot automatiza identificarea algoritmilor și clasificarea cererilor tehnice, asistând în alegerea metodelor criptanalitice adecvate, conform tiparelor datelor \parencite{kim2024_cryptanalysis,hu2025_cnn}. Integrarea tehnicilor de RAG permite asistenților AI să furnizeze recomandări fundamentate pe literatura de specialitate și standarde, susținând tranziția către algoritmi post-cuantici și reducând riscul de halucinații \parencite{dong2025_chatiot}. În evaluarea parolelor, se folosesc modele neuronale antrenate pe parole comune combinate cu estimatori euristici (zxcvbn) și baze de date publice precum HIBP, ceea ce îmbunătățește precizia scorării și a analizei de risc \parencite{mo2025_password_strength,wheeler2016_zxcvbn}. Generarea de scenarii și date sintetice prin modele generative facilitează testarea rezilienței sistemelor, iar sistemele de detecție bazate pe învățare profundă pot identifica comportamente suspecte în trafic sau log-uri \parencite{xu2025_ids}. Criptografia contribuie la protecția AI prin tehnici precum criptarea omomorfă, calculul multipartit securizat sau privacy diferențială, folosite pentru protejarea datelor și a modelelor în timpul antrenării sau inferenței \parencite{lee2022_ppml}. Pentru integritatea modelelor, se propun watermark-uri digitale și semnături criptografice atașate acestora. Totuși, AI introduce riscuri semnificative, precum prompt injection sau exfiltrarea secretelor stocate, motiv pentru care sunt necesare guardrails, monitorizare și evaluare continuă a comportamentului modelului \parencite{li2025_llm_security}.

\chapter{Soluția Propusă}
\label{ch:proposed-solution}

\section{Arhitectura generală}
Arhitectura generală a aplicației este formată dintr-un set de microservicii containerizate, fiecare reprezentând o componentă cheie în funcționarea soluției. Pentru a face posibilă existența unei soluții scalabile, ușor de întreținut și modulare, este nevoie de o orchestrare eficientă a acestor microservicii, prin componente auxiliare.
Astfel că am identificat următoarele componente principale:
\begin{itemize}
    \item Frontend: care poate să fie un UI web, sau o interfață din linie de comandă.
    \item Backend: care gestionează logica de business, comunicarea între componente și expune API-urile necesare.
    \item Protection Layer: asigură securitatea datelor în tranzit, soluția exactă de securitate fiind aleasă ulterior.
    \item Baza de date: pentru stocarea persistentă a datelor relevante.
    \item Orchestrator: care asigură un flow de lucru bine înrădăcinat semantic, pentru evitarea și minimizarea erorilor.
    \item Decision Pipeline: un pipeline care să asigure extragerea intenției și a entităților relevante din cererile utilizatorilor.
    \item AI Agents Pool: un set de agenți AI specializați, fiecare cu rolul său specific în cadrul soluției.
\end{itemize}
Fiecare dintre agenți la rândul lui poate fi văzut ca un microserviciu containerizat, care poate fi dezvoltat, testat și implementat independent. Aceștia pot comunica între ei prin intermediul unor API-uri bine definite, permițând astfel o flexibilitate și o scalabilitate sporită a întregii soluții.
În viziunea mea prezentă, partea de security ar trebui să fie implementată la nivelul microserviciului, și partea de rate-limit și monitorizare a traficului între componente, la nivel centralizat.

\begin{figure}[H]
    \centering
    \includegraphics[height=0.69\textheight]{diagrama_arhitectura_generala.pdf}
    \caption{Arhitectura generală a solutiei propuse.}
    \label{fig:arch-general}
\end{figure}

\begin{figure}[H]
    \centering
    \includegraphics[width=0.9\textwidth]{diagrama_agenti.pdf}
    \caption{Pool-ul de agenți AI specializați.}
    \label{fig:ai-agents}
\end{figure}

\section{Metodologie și justificarea alegerilor}

\subsection{Selecția tehnologiilor}

\subsection{Principii de rețelistică}

\subsection{Principii de securitate}

\section{Descrierea componentelor}

\subsection{Modul AI}

\subsection{Modul criptografic}

\subsection{Orchestrator și interfețe}

\chapter{Implementare}
\label{ch:implementation}

\section{Mediu de dezvoltare}

\section{Framework-uri și librării}

\section{Containerizare și infrastructură}

\section{Măsuri de securitate aplicate}

\section{Automatizare}

\chapter{Testare și Evaluare}
\label{ch:testing}

\section{Metodologia de testare}

\subsection{Teste unitare}

\subsection{Teste de integrare}

\subsection{Teste de performanță}

\section{Metrici și criterii de evaluare}

\section{Validarea rezultatelor}

\chapter{Rezultate și Discuții}
\label{ch:results}

\section{Benchmark-uri comparative}

\section{Analiza performanței}

\section{Beneficii și limitări}

\section{Direcții de îmbunătățire}

\chapter{Concluzii}
\label{ch:conclusions}

\section{Gradul de atingere a obiectivelor}

\section{Contribuții}

\section{Impact și dezvoltări viitoare}

\chapter*{Bibliografie}
\addcontentsline{toc}{chapter}{Bibliografie}

\printbibliography[heading=none]

\chapter{Anexe}
\label{ch:annexes}

\section{Listă de diagrame suplimentare}

\section{Configurații tehnice}

\section{Scripturi și manual de utilizare}

\end{document}
