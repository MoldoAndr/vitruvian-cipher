\documentclass[../../main.tex]{subfiles}

\begin{document}

\chapter{Testare și Evaluare}
\label{ch:testing}

\section{Metodologia de testare}
{\color{gray}
Descrie strategia generală de testare și criteriile utilizate pentru a valida soluția.
}

\subsection{Teste unitare}
{\color{gray}
Prezintă acoperirea testelor unitare, instrumentele folosite și exemple relevante.
}

\subsection{Teste de integrare}
{\color{gray}
Detaliază scenariile de integrare și modul în care componentele sunt validate împreună.
}

\subsection{Teste de performanță}
{\color{gray}
Specifică metodologiile și instrumentele folosite pentru măsurarea performanței și scalabilității.
}

\section{Metrici și criterii de evaluare}
{\color{gray}
Definește indicatorii cantitativi și calitativi utilizați în analizarea rezultatelor testelor.
}

\section{Validarea rezultatelor}
{\color{gray}
Analizează rezultatele obținute în raport cu obiectivele propuse și discută implicațiile.
}

\end{document}
