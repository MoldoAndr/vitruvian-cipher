\documentclass[../../main.tex]{subfiles}

\begin{document}

\chapter{Introducere}
\label{ch:introduction}

\section{Context general}
{\color{gray}
Prezintă pe scurt evoluția criptografiei moderne și modul în care tehnologiile de inteligență artificială influențează securitatea cibernetică.
}

\section{Problema abordată}
{\color{gray}
Descrie problema principală a lucrării, pornind de la nevoile identificate în domeniul securității cibernetice asistate de AI.
}

\section{Scopul lucrării}
{\color{gray}
Formulează clar scopul general al proiectului și rezultatul urmărit prin dezvoltarea utilitarului criptografic asistat de AI.
}

\section{Obiective specifice}
{\color{gray}
Prezintă obiectivele concrete ce trebuie îndeplinite pentru atingerea scopului.
}

\begin{itemize}
    \item {\color{gray}O1 -- Definirea cadrului teoretic și a cerințelor specifice aplicației.}
    \item {\color{gray}O2 -- Proiectarea arhitecturii soluției propuse.}
    \item {\color{gray}O3 -- Implementarea componentelor software și integrarea acestora.}
    \item {\color{gray}O4 -- Validarea soluției prin scenarii de testare reprezentative.}
\end{itemize}

\section{Structura lucrării}
{\color{gray}
Include o prezentare succintă a fiecărui capitol pentru a ghida cititorul prin document.
}

\end{document}
