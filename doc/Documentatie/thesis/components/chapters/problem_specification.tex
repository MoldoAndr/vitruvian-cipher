\documentclass[../../main.tex]{subfiles}

\begin{document}

\chapter{Definirea Problemei și Specificații}
\label{ch:problem-spec}

\section{Problema abordată}
{\color{gray}
Precizează limitele sistemelor curente și explică problema concretă pe care o adresează lucrarea.
}

\section{Cerințe funcționale}
{\color{gray}
Listează funcționalitățile pe care soluția trebuie să le îndeplinească pentru a răspunde problemei definită anterior.
}

\section{Cerințe nefuncționale}
{\color{gray}
Documentează cerințele de performanță, scalabilitate, securitate, disponibilitate și mentenanță.
}

\section{Model de atac și scenarii de amenințări}
{\color{gray}
Identifică actorii rău intenționați, capabilitățile lor și scenariile de atac ce trebuie contracarate.
}

\section{Obiective aplicate soluției}
{\color{gray}
Reformulează obiectivele generale într-un set de livrabile verificabile pentru soluția implementată.
}

\end{document}
