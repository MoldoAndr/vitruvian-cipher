\documentclass[../../main.tex]{subfiles}

\begin{document}

\chapter{Fundamente Teoretice}
\label{ch:foundations}

\section{Elemente de bază ale criptografiei}
{\color{gray}
Prezintă noțiunile fundamentale legate de criptografie, clasificările principale și conceptele de bază necesare pentru înțelegerea soluției propuse.
}

\subsection{Definiții și clasificare}
{\color{gray}
Include definițiile pentru criptografie simetrică, criptografie asimetrică, funcții hash și alte primitive relevante. Discută diferențele între acestea și cazurile tipice de utilizare.
}

\subsection{Standardizare}
{\color{gray}
Enumeră standardele și organismele relevante (NIST, IEEE, RFC, GDPR, standarde militare) și menționează rolul lor în conformarea soluției propuse.
}

\section{Elemente fundamentale de inteligență artificială}
{\color{gray}
Descrie conceptele cheie din domeniul inteligenței artificiale care vor fi folosite în lucrare.
}

\subsection{Paradigme de învățare}
{\color{gray}
Discută despre învățarea supravegheată, nesupravegheată și prin întărire, raportându-te la scenariul lucrării.
}

\subsection{Modele generative și transformere}
{\color{gray}
Introduce arhitecturile Transformer, modelele generative moderne și implicațiile lor în securitatea cibernetică.
}

\subsection{GAN și LLM în securitate}
{\color{gray}
Prezintă modul în care GAN și LLM sunt utilizate în detecția și generarea de conținut relevant pentru securitate, incluzând exemple recente.
}

\section{Conexiunea AI--Criptografie}
{\color{gray}
Leagă componentele AI și criptografie pentru a fundamenta soluția propusă.
}

\subsection{Domenii de aplicare}
{\color{gray}
Analizează scenarii precum cryptoanalysis asistată de AI, generarea de parole, detecția anomaliilor sau automatizarea proceselor de securitate.
}

\subsection{Limitări curente}
{\color{gray}
Identifică limitările actuale ale abordărilor combinate AI--criptografie și explică modul în care soluția propusă răspunde acestor provocări.
}

\end{document}
