\documentclass[../../main.tex]{subfiles}

\begin{document}

\chapter{Soluția Propusă}
\label{ch:proposed-solution}

\section{Arhitectura generală}
{\color{gray}
Descrie arhitectura macro a sistemului și include o diagramă care evidențiază principalele componente și fluxuri de date.
}

\section{Metodologie și justificarea alegerilor}
{\color{gray}
Explică abordarea metodologică, pașii parcurși și motivele pentru alegerea tehnologiilor și a modelelor AI.
}

\subsection{Selecția tehnologiilor}
{\color{gray}
Detaliază limbajele, cadrele și infrastructura utilizate și argumentează selecția lor.
}

\subsection{Modele AI utilizate}
{\color{gray}
Prezintă modelele AI integrate, modul în care sunt antrenate, configurate și orchestrate.
}

\subsection{Principii de securitate}
{\color{gray}
Descrie măsurile de securitate și mecanismele de protecție integrate încă din etapa de proiectare.
}

\section{Descrierea componentelor}
{\color{gray}
Analizează fiecare componentă a soluției folosind paragrafe dedicate ce includ rolul, funcționarea, diagrama și justificarea tehnologică.
}

\subsection{Modul AI}
{\color{gray}
Detaliază agentul AI, capabilitățile, datele utilizate și modul de interacțiune cu celelalte componente.
}

\subsection{Modul criptografic}
{\color{gray}
Descrie serviciile criptografice, protocoalele implementate și modul de integrare cu orchestrarea AI.
}

\subsection{Orchestrator și interfețe}
{\color{gray}
Prezintă modul de coordonare între componente, fluxurile de lucru și interfețele expuse (CLI, API, UI).
}

\end{document}
