\documentclass[oneside]{report}

% Pachete de bază (alese din cadrul șablonului robust)
\usepackage[utf8]{inputenc}
\usepackage[a4paper, top=2cm, bottom=2cm, left=2.5cm, right=2.5cm]{geometry}
\usepackage[
    backend=bibtex,
    style=numeric,
    bibencoding=ascii,
    sorting=anyt
]{biblatex}
\usepackage{caption}
\usepackage{datetime}
\usepackage{etoolbox}
\usepackage{fancyhdr}
\usepackage{graphics}
\usepackage{graphicx}
\usepackage{hyperref}
\usepackage{lastpage}
\usepackage{lscape}
\usepackage{subcaption}
\usepackage{tablefootnote}
\usepackage[titles]{tocloft}
\usepackage{titlesec}
\usepackage{xcolor}

\hypersetup{hidelinks}
\graphicspath{{images/}}
\addbibresource{bibliography.bib}

% Formatare titluri și liste
\setlength{\cftfigindent}{0em}
\setlength{\cfttabindent}{0em}
\titleformat{\chapter}[display]{\bfseries\huge}{Capitolul \thechapter:}{0pt}{}[]

% Stiluri pentru antet/subsol
\setlength{\headheight}{12pt}
\setlength{\headsep}{10pt}
\setlength{\footskip}{14pt}
\renewcommand{\headrulewidth}{1pt}
\renewcommand{\footrulewidth}{1pt}
\fancypagestyle{front}{
  \fancyhf{}
  \fancyhead[C]{\small NECLASIFICAT}
  \fancyfoot[C]{\small NECLASIFICAT\\\thepage}
}
\pagestyle{front}
\fancypagestyle{plain}{
  \fancyhf{}
  \fancyhead[C]{\small NECLASIFICAT}
  \fancyfoot[C]{\small NECLASIFICAT\\\thepage\ din \pageref*{LastPage}}
}
\newenvironment{frontmatter}{\pagenumbering{roman}}{\newpage \pagenumbering{arabic}}
\AtBeginDocument{\addtocontents{toc}{\protect\thispagestyle{front}}}
\addtocontents{lof}{\protect\thispagestyle{front}}

% Comenzi auxiliare
\newcommand{\textbfit}[1]{\textbf{\textit{#1}}}
\newenvironment{codeenv}{\fontfamily{cmtt}\selectfont}{\par}
\DeclareTextFontCommand{\inlinecodenonboxed}{\codeenv}
\newcommand{\inlinecode}[1]{\mbox{\inlinecodenonboxed{#1}}}
\newenvironment{abbreviations}{\begin{list}{}{\renewcommand{\makelabel}{\abbrlabel}}}{\end{list}}
\newcommand{\abbrlabel}[1]{\makebox[3cm][l]{\textbf{#1}\ \dotfill}}

% Configurare metadate (editabil rapid)
\newcommand{\detaildate}{\color{gray}2024}
\newcommand{\detailcountry}{România}
\newcommand{\detailcity}{București}
\newcommand{\detailparentinstitution}{Ministerul Apărării Naționale}
\newcommand{\detailuniversity}{Academia Tehnică Militară ``\textit{Ferdinand I}''}
\newcommand{\detailfaculty}{Facultatea de Sisteme Informatice și Securitate Cibernetică}
\newcommand{\detailspecialization}{\color{gray}Specializarea absolvită}
\newcommand{\detailtitle}{\color{gray}Titlul lucrării}
\newcommand{\detailauthor}{\color{gray}Grad Prenume NUME}
\newcommand{\detailadviser}{\color{gray}Grad Prenume NUME}

\begin{document}

\begin{frontmatter}

% Pagina de titlu
\begin{titlepage}
    \begin{center}

        \textbf{\detailcountry} \\
        \textbf{\detailparentinstitution} \\
        \textbf{\detailuniversity} \\

        \vspace*{0.5cm}

        \textbf{\detailfaculty} \\
        \textbf{\detailspecialization} \\

        \vspace*{1.5cm}

        \includegraphics[width=0.4\textwidth]{ATM.png}

        \vspace*{1.5cm}

        \large{\textbf{\detailtitle}}

        \vspace*{0.5cm}

        \normalsize
        \begin{tabular*}{\textwidth}{l@{\extracolsep{\fill}}r}
            \textbf{Coordonator Științific} & \textbf{Absolvent}\\
            \footnotesize \detailadviser & \footnotesize \detailauthor
        \end{tabular*}

        \vspace*{3cm}

        \vspace*{1.5cm}
        \textbf{\detailcity} \\
        \textbf{\detaildate}

    \end{center}
\end{titlepage}

% Paginare pentru frontmatter
\setcounter{page}{1}

% Abstract (RO)
\chapter*{Abstract}
\thispagestyle{front}
{\color{gray}
Rezumatul în limba română trebuie să reflecte fidel conținutul lucrării: context, obiective, metodologie, rezultate și concluzii. Păstrează o dimensiune similară cu versiunea în limba engleză și evită informații care nu apar în capitole.
}

\bigskip

{\color{gray}
    \textbf{Cuvinte cheie:} inteligență artificială; criptografie; securitate cibernetică; evaluare
}

\newpage

% Abstract (EN)
\chapter*{Abstract (EN)}
\thispagestyle{front}
{\color{gray}
Summarize the thesis in English, mirroring the Romanian version: context, objectives, methodology, results, and conclusions. Keep the length aligned with the Romanian abstract.
}

\bigskip

{\color{gray}
    \textbf{Keywords:} artificial intelligence; cryptography; cybersecurity; evaluation
}

\newpage

% Cuprins + liste
\setcounter{secnumdepth}{3}
\phantomsection
\tableofcontents
\newpage
\pagestyle{front}\listoffigures
\newpage
\pagestyle{front}\listoftables
\newpage

% Lista de abrevieri
\chapter*{Listă de Abrevieri}
\thispagestyle{front}
\begin{abbreviations}
    {\color{gray}
    \item[AI] artificial intelligence
    \item[ML] machine learning
    \item[PKI] public key infrastructure
    \item[RAG] retrieval-augmented generation
    \item[TLS] transport layer security
    }
\end{abbreviations}
\newpage

\end{frontmatter}

% -------------------------
% Conținut capitole
% -------------------------
\pagestyle{plain}

\chapter{Introducere}
\label{ch:introduction}

\section{Context general}
{\color{gray}
Prezintă pe scurt evoluția criptografiei moderne și modul în care tehnologiile de inteligență artificială influențează securitatea cibernetică.
}

\section{Problema abordată}
{\color{gray}
Descrie problema principală a lucrării, pornind de la nevoile identificate în domeniul securității cibernetice asistate de AI.
}

\section{Scopul lucrării}
{\color{gray}
Formulează clar scopul general al proiectului și rezultatul urmărit prin dezvoltarea utilitarului criptografic asistat de AI.
}

\section{Obiective specifice}
{\color{gray}
Prezintă obiectivele concrete ce trebuie îndeplinite pentru atingerea scopului.
}
\begin{itemize}
    \item {\color{gray}O1 -- Definirea cadrului teoretic și a cerințelor specifice aplicației.}
    \item {\color{gray}O2 -- Proiectarea arhitecturii soluției propuse.}
    \item {\color{gray}O3 -- Implementarea componentelor software și integrarea acestora.}
    \item {\color{gray}O4 -- Validarea soluției prin scenarii de testare reprezentative.}
\end{itemize}

\section{Structura lucrării}
{\color{gray}
Include o prezentare succintă a fiecărui capitol pentru a ghida cititorul prin document.
}

\chapter{Fundamente Teoretice}
\label{ch:foundations}

\section{Elemente de bază ale criptografiei}
{\color{gray}
Prezintă noțiunile fundamentale legate de criptografie, clasificările principale și conceptele de bază necesare pentru înțelegerea soluției propuse.
}

\subsection{Definiții și clasificare}
{\color{gray}
Include definițiile pentru criptografie simetrică, criptografie asimetrică, funcții hash și alte primitive relevante. Discută diferențele între acestea și cazurile tipice de utilizare.
}

\subsection{Standardizare}
{\color{gray}
Enumeră standardele și organismele relevante (NIST, IEEE, RFC, GDPR, standarde militare) și menționează rolul lor în conformarea soluției propuse.
}

\section{Elemente fundamentale de inteligență artificială}
{\color{gray}
Descrie conceptele cheie din domeniul inteligenței artificiale care vor fi folosite în lucrare.
}

\subsection{Paradigme de învățare}
{\color{gray}
Discută despre învățarea supravegheată, nesupravegheată și prin întărire, raportându-te la scenariul lucrării.
}

\subsection{Modele generative și transformere}
{\color{gray}
Introduce arhitecturile Transformer, modelele generative moderne și implicațiile lor în securitatea cibernetică.
}

\subsection{GAN și LLM în securitate}
{\color{gray}
Prezintă modul în care GAN și LLM sunt utilizate în detecția și generarea de conținut relevant pentru securitate, incluzând exemple recente.
}

\section{Conexiunea AI--Criptografie}
{\color{gray}
Leagă componentele AI și criptografie pentru a fundamenta soluția propusă.
}

\subsection{Domenii de aplicare}
{\color{gray}
Analizează scenarii precum cryptoanalysis asistată de AI, generarea de parole, detecția anomaliilor sau automatizarea proceselor de securitate.
}

\subsection{Limitări curente}
{\color{gray}
Identifică limitările actuale ale abordărilor combinate AI--criptografie și explică modul în care soluția propusă răspunde acestor provocări.
}

\chapter{Stadiul Actual al Soluțiilor Existente}
\label{ch:state-of-art}

\section{Analiză comparativă}
{\color{gray}
Prezintă soluțiile existente (academice și comerciale), avantajele și limitările lor, precum și punctele comune cu proiectul tău.
}

\section{Cercetare academică relevantă}
{\color{gray}
Sintetizează lucrări recente din domeniul AI și criptografie, evidențiind contribuțiile și lipsurile.
}

\section{Motivarea soluției proprii}
{\color{gray}
Conectează limitările identificate la necesitatea soluției propuse și formulează argumentele principale.
}

\chapter{Definirea Problemei și Specificații}
\label{ch:problem-specification}

\section{Problema abordată}
{\color{gray}
Descrie clar problema practică pe care o rezolvă lucrarea, incluzând contextul operațional.
}

\section{Cerințe funcționale}
{\color{gray}
Listează funcționalitățile așteptate, preferabil sub formă de bullet list sau tabel.
}

\section{Cerințe nefuncționale}
{\color{gray}
Evidențiază cerințe de performanță, scalabilitate, securitate, disponibilitate și monitorizare.
}

\section{Model de atac / scenariu de amenințări}
{\color{gray}
Prezintă potențialii adversari, capabilitățile lor și vectorii de atac relevanți.
}

\section{Obiective aplicate soluției}
{\color{gray}
Reformulează obiectivele în raport cu cerințele de mai sus și modul în care vor fi validate.
}

\chapter{Soluția Propusă}
\label{ch:proposed-solution}

\section{Arhitectura generală}
{\color{gray}
Descrie arhitectura macro a sistemului și include o diagramă care evidențiază principalele componente și fluxuri de date.
}

\section{Metodologie și justificarea alegerilor}
{\color{gray}
Explică abordarea metodologică, pașii parcurși și motivele pentru alegerea tehnologiilor și a modelelor AI.
}

\subsection{Selecția tehnologiilor}
{\color{gray}
Detaliază limbajele, cadrele și infrastructura utilizate și argumentează selecția lor.
}

\subsection{Modele AI utilizate}
{\color{gray}
Prezintă modelele AI integrate, modul în care sunt antrenate, configurate și orchestrate.
}

\subsection{Principii de securitate}
{\color{gray}
Descrie măsurile de securitate și mecanismele de protecție integrate încă din etapa de proiectare.
}

\section{Descrierea componentelor}
{\color{gray}
Analizează fiecare componentă a soluției folosind paragrafe dedicate ce includ rolul, funcționarea, diagrama și justificarea tehnologică.
}

\subsection{Modul AI}
{\color{gray}
Detaliază agentul AI, capabilitățile, datele utilizate și modul de interacțiune cu celelalte componente.
}

\subsection{Modul criptografic}
{\color{gray}
Descrie serviciile criptografice, protocoalele implementate și modul de integrare cu orchestrarea AI.
}

\subsection{Orchestrator și interfețe}
{\color{gray}
Prezintă modul de coordonare între componente, fluxurile de lucru și interfețele expuse (CLI, API, UI).
}

\chapter{Implementare}
\label{ch:implementation}

\section{Mediu de dezvoltare}
{\color{gray}
Detaliază uneltele de dezvoltare, sistemul de operare, dependințele și configurațiile cheie.
}

\section{Framework-uri și librării}
{\color{gray}
Listează și descrie framework-urile, librăriile și modelele AI utilizate, inclusiv versiunile.
}

\section{Containerizare și infrastructură}
{\color{gray}
Prezintă modul în care aplicația este containerizată și orchestrată (Docker, docker-compose, Kubernetes etc.).
}

\section{Măsuri de securitate aplicate}
{\color{gray}
Enumeră controalele de securitate implementate (hardening, politici, rotație de chei, audit).
}

\section{Automatizare}
{\color{gray}
Descrie scripturile, pipeline-urile CI/CD sau orice automatizare care susține proiectul.
}

\chapter{Testare și Evaluare}
\label{ch:testing}

\section{Metodologia de testare}
{\color{gray}
Descrie strategia generală de testare și criteriile utilizate pentru a valida soluția.
}

\subsection{Teste unitare}
{\color{gray}
Prezintă acoperirea testelor unitare, instrumentele folosite și exemple relevante.
}

\subsection{Teste de integrare}
{\color{gray}
Detaliază scenariile de integrare și modul în care componentele sunt validate împreună.
}

\subsection{Teste de performanță}
{\color{gray}
Specifică metodologiile și instrumentele folosite pentru măsurarea performanței și scalabilității.
}

\section{Metrici și criterii de evaluare}
{\color{gray}
Definește indicatorii cantitativi și calitativi utilizați în analizarea rezultatelor testelor.
}

\section{Validarea rezultatelor}
{\color{gray}
Analizează rezultatele obținute în raport cu obiectivele propuse și discută implicațiile.
}

\chapter{Rezultate și Discuții}
\label{ch:results}

\section{Benchmark-uri comparative}
{\color{gray}
Prezintă rezultate numerice și comparații cu soluții similare sau baseline-uri.
}

\section{Analiza performanței}
{\color{gray}
Evaluează latența, throughput-ul, utilizarea resurselor și scalabilitatea.
}

\section{Beneficii și limitări}
{\color{gray}
Rezumă beneficiile soluției și limitele identificate în urma testării.
}

\section{Direcții de îmbunătățire}
{\color{gray}
Propune optimizări și extensii viitoare.
}

\chapter{Concluzii}
\label{ch:conclusions}

\section{Gradul de atingere a obiectivelor}
{\color{gray}
Recapitulează modul în care au fost îndeplinite obiectivele definite în introducere.
}

\section{Contribuții}
{\color{gray}
Enumeră contribuțiile teoretice și practice ale lucrării.
}

\section{Impact și dezvoltări viitoare}
{\color{gray}
Evaluează impactul potențial și prezintă direcții de continuare a cercetării.
}

\chapter*{Bibliografie}
\addcontentsline{toc}{chapter}{Bibliografie}
{\color{gray}
Folosește \verb|\cite| pe parcursul textului; sursele se vor lista automat mai jos.
}

\printbibliography[heading=none]

\chapter{Anexe}
\label{ch:annexes}

\section{Listă de diagrame suplimentare}
{\color{gray}
Include diagrame detaliate (de exemplu, PlantUML, diagrame de secvență sau arhitecturi alternative) care completează prezentarea din capitolele principale.
}

\section{Configurații tehnice}
{\color{gray}
Prezintă fișiere de configurare relevante (de exemplu, \inlinecode{docker-compose.yml}, fișiere de mediu, politici de securitate) și explică modul în care acestea sunt utilizate.
}

\section{Scripturi și manual de utilizare}
{\color{gray}
Include scripturile auxiliare și un ghid succint de utilizare a aplicației pentru diferitele roluri implicate.
}

\end{document}
