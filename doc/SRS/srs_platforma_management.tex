\documentclass[12pt,a4paper]{article}

% Pachete necesare
\usepackage[utf8]{inputenc}
\usepackage[romanian]{babel}
\usepackage{graphicx}
\usepackage{geometry}
\usepackage{fancyhdr}
\usepackage{titlesec}
\usepackage{enumitem}
\usepackage{hyperref}
\usepackage{longtable}
\usepackage{array}
\usepackage{booktabs}
\usepackage{xcolor}
\usepackage{float}
\usepackage{colortbl}
\usepackage{tikz}

% Configurare pagină
\geometry{
	left=2.5cm,
	right=2cm,
	top=2.5cm,
	bottom=2.5cm
}

% Configurare hyperlink
\hypersetup{
	colorlinks=true,
	linkcolor=blue,
	filecolor=magenta,      
	urlcolor=cyan,
	pdftitle={SRS - Platforma Management Universitar},
	pdfauthor={Grupa Apahidean}
}

% Header și footer
\pagestyle{fancy}
\fancyhf{}
\fancyhead[L]{\small SRS - Platformă Management Universitar}
\fancyhead[R]{\small v1.0 - Noiembrie 2025}
\fancyfoot[C]{\thepage}

% Definire culori
\definecolor{headcolor}{RGB}{0,102,204}            % Albastru pentru titluri
\definecolor{tableheader}{RGB}{220,235,247}    % Albastru deschis pentru tabele

% Formatare titluri secțiuni
\titleformat{\section}
{\normalfont\Large\bfseries\color{headcolor}}{\thesection}{1em}{}

\titleformat{\subsection}
{\normalfont\large\bfseries\color{headcolor}}{\thesubsection}{1em}{}

\titleformat{\subsubsection}
{\normalfont\normalsize\bfseries}{\thesubsubsection}{1em}{}

% Comenzi personalizate pentru cerințe
\newcommand{\reqshall}{\textbf{TREBUIE }}
\newcommand{\reqshould}{\textbf{AR TREBUI }}
\newcommand{\reqmay}{\textbf{POATE }}

\begin{document}
	
	% ========== PAGINA DE TITLU ==========
	\begin{titlepage}
		\begin{center}
			
			\textbf{România} \\
			\textbf{Ministerul Apărării Naționale} \\
			\textbf{Academia Tehnică Militară „Ferdinand I"} \\
			
			\vspace*{0.3cm}
			
			\textbf{Facultatea de Sisteme Informatice și Securitate Cibernetică} \\
			Calculatoare și sisteme informatice pentru apărare și securitate națională \\
			
			\vspace*{1cm}
			
			% Logo ATM
			\includegraphics[width=0.35\textwidth]{logo_atm.png}
			\vspace*{1.5cm}
			
			\large{\textbf{Platformă de Management Universitar}}\\
			\vspace*{0.2cm}
			\normalsize{Documentul de Specificare a Cerințelor Software}
			
			\vspace*{0.5cm}
			
			\normalsize
			Disciplina: Ingineria Programării
			
			\vspace*{0.8cm}
			
			\small
			\begin{tabular*}{\textwidth}{l@{\extracolsep{\fill}}r}
				\textbf{Profesor Coordonator} & \textbf{Echipa de Proiect}\\
				Slt.Ing.(Asist.Univ) Leancă Răzvan & Sd.Sg.Maj. Apahidean Luca \\
				& Sd.Plt.Maj. Leu Cătălin \\
				& Sd.Sg.Maj. Dumitru Estera \\
				& Sd.Sg.Maj. Shanab Bianca \\
				& Sd.Sg.Maj. Andrei Diana \\
				& Sd.Sg.Maj. Moldovan Andrei \\
				& Sd.Sg.Maj. Buontempo Raul \\
				& Sd.Sg.Maj. Milea Alexandru
			\end{tabular*}
			\normalsize
			
			\vspace*{1.5cm}
			
			București \\
			Noiembrie 2025
			
		\end{center}
	\end{titlepage}
	
	% ========== CONTROL DOCUMENT ==========
	\newpage
	\section*{Control Document}
	
	\begin{table}[H]
		\centering
		\begin{tabular}{|p{4cm}|p{10cm}|}
			\hline
			\textbf{Titlu} & Specificația Cerințelor Software pentru Platforma de Management Universitar \\
			\hline
			\textbf{Data} & 4 Noiembrie 2025 \\
			\hline
			\textbf{Status} & Draft \\
			\hline
			\textbf{Versiune} & 1.0 \\
			\hline
			\textbf{Pregătit pentru} & Academia Tehnică Militară „Ferdinand I" \\
			\hline
			\textbf{Referință} & SRS\_PMU\_V1.0\_Nov\_2025 \\
			\hline
		\end{tabular}
	\end{table}
	
	\subsection*{Disclaimer}
	Acest document este pregătit în scopuri academice pentru proiectul de laborator la disciplina Ingineria Programării al grupei de studenți menționați pe pagina de titlu. Documentul conține specificațiile detaliate ale cerințelor software pentru Platforma de Management Universitar.
	
	% ========== CUPRINS ==========
	\newpage
	\tableofcontents
	
	% ========== GLOSAR ==========
	\newpage
	\section*{Glosar}
	\addcontentsline{toc}{section}{Glosar}
	
	\begin{table}[H]
		\centering
		\begin{tabular}{|p{3cm}|p{11cm}|}
			\hline
			\rowcolor{tableheader}
			\textbf{Termen} & \textbf{Definiție} \\
			\hline
			API & Application Programming Interface \\
			\hline
			RBAC & Role-Based Access Control \\
			\hline
			SRS & Software Requirements Specification \\
			\hline
			CRUD & Create, Read, Update, Delete \\
			\hline
			CSV & Comma-Separated Values \\
			\hline
			PDF & Portable Document Format \\
			\hline
			CNP & Cod Numeric Personal \\
			\hline
			JWT & JSON Web Token \\
			\hline
			2FA & 2 Factor Authentication\\
			\hline
			SQL & Structured Query Language \\
			\hline
			XSS & Cross-Site Scripting \\
			\hline
			CSRF & Cross-Site Request Forgery \\
			\hline
			TTL & Time to Live \\
			\hline
		\end{tabular}
	\end{table}
	
	% ========== INTRODUCERE ==========
	\newpage
	\section{Introducere}
	
	\subsection{Scopul Documentului}
	
	Acest document SRS oferă o descriere completă a Platformei de Management Universitar destinate Academiei Tehnice Militare „Ferdinand I". Documentul detaliază cerințele funcționale și non-funcționale ale platformei, care are ca scop optimizarea proceselor administrative și academice ale instituției.
	
	Platforma va facilita managementul eficient al:
	\begin{itemize}[noitemsep]
		\item Utilizatorilor și rolurilor (studenți, profesori, secretariat, administratori)
		\item Planurilor de învățământ și disciplinelor
		\item Orarului și planificării examenelor
		\item Catalogului și notelor academice
		\item Comunicării și notificărilor
	\end{itemize}
	
	Acest SRS va servi drept bază pentru fazele ulterioare de proiectare, dezvoltare și testare ale sistemului, asigurând că persoanele implicate au o înțelegere clară a funcționalităților și operațiunilor sistemului.
	
	\subsection{Convenții Document}
	
	Acest document urmează următoarele convenții:
	
	\begin{table}[H]
		\centering
		\begin{tabular}{|p{3cm}|p{11cm}|}
			\hline
			\rowcolor{tableheader}
			\textbf{Termen} & \textbf{Descriere} \\
			\hline
			\reqshall & Indică o cerință obligatorie care trebuie implementată în faza curentă de dezvoltare. \\
			\hline
			\reqshould & Indică o cerință recomandată care ar trebui implementată pentru funcționalitate optimă. \\
			\hline
			\reqmay & Indică o cerință opțională care poate fi implementată în funcție de resurse și prioritizare. \\
			\hline
		\end{tabular}
	\end{table}
	
	Cerințele sunt categorisite astfel:
	
	\begin{table}[H]
		\centering
		\begin{tabular}{|p{4cm}|p{10cm}|}
			\hline
			\rowcolor{tableheader}
			\textbf{Număr Cerință} & \textbf{Descriere} \\
			\hline
			FR-XXX & Cerințe Funcționale \\
			\hline
			NFR-XXX & Cerințe Non-Funcționale \\
			\hline
			SR-XXX & Cerințe de Securitate \\
			\hline
		\end{tabular}
	\end{table}
	
	\subsection{Audiență Intenționată}
	
	Acest document este destinat următoarelor persoane:
	
	\begin{table}[H]
		\centering
		\begin{tabular}{|p{4cm}|p{10cm}|}
			\hline
			\rowcolor{tableheader}
			\textbf{Persoană} & \textbf{Rol} \\
			\hline
			Academia Tehnică Militară & Conducerea instituției care va aproba și exploata sistemul \\
			\hline
			Echipa de dezvoltare & Dezvoltatorii care vor implementa sistemul \\
			\hline
			Administratori sistem & Personalul tehnic care va întreține și administra platforma \\
			\hline
			Secretariat & Utilizatori care vor gestiona datele academice și administrative \\
			\hline
			Cadre didactice & Profesori care vor utiliza sistemul pentru activități didactice \\
			\hline
			Studenți & Utilizatori finali care vor accesa informațiile academice \\
			\hline
			Evaluatori & Persoane care vor evalua conformitatea implementării \\
			\hline
		\end{tabular}
	\end{table}
	
	\subsection{Cadrul Proiectului}
	
	Platforma de Management Universitar este un sistem informatic integrat destinat automatizării și optimizării proceselor academice și administrative din cadrul Academiei Tehnice Militare „Ferdinand I".
	
	\subsubsection{În Cadrul Proiectului}
	
	Platforma va include următoarele componente principale:
	
	\begin{enumerate}[noitemsep]
		\item Modul de Management Utilizatori
		\item Modul Secretariat
		\item Modul Orar
		\item Modul Catalog
		\item Modul Planificare Examene
		\item Modul Notificări
	\end{enumerate}
	
	\subsubsection{În Afara Domeniului Proiectului}
	
	Următoarele aspecte sunt excluse explicit din acest proiect:
	
	\begin{enumerate}[noitemsep]
		\item Sistem de plăți online pentru taxe studențești
		\item Portal de e-learning sau platformă educațională (e.g. Moodle)
		\item Sistem de bibliotecă digitală
		\item Aplicație mobilă nativă (iOS/Android) - se va implementa versiune web responsive
		\item Sistem de management al căminelor și cantinelor
		\item Portal pentru alumni sau absolvenți
		\item Sistem de management al cercetării științifice
	\end{enumerate}
	
	% ========== DESCRIERE GENERALĂ ==========
	\newpage
	\section{Descriere Generală}
	
	\subsection{Perspectiva Produsului}
	
	Platforma de Management Universitar este un sistem dezvoltat specific pentru Academia Tehnică Militară „Ferdinand I". Sistemul va opera ca o aplicație web centralizată, accesibilă prin browsere standard, fără a necesita instalarea de software client.
	
	\subsection{Funcționalitățile Produsului}
	
	Platforma de Management Universitar va oferi următoarele funcționalități majore:
	
	\begin{enumerate}
		\item \textbf{Autentificare și Managementul Accesului}
		\begin{itemize}[noitemsep]
			\item Autentificare securizată cu email și parolă
			\item Sistem RBAC
			\item Gestionarea sesiunilor utilizator
		\end{itemize}
		
		\item \textbf{Administrare Utilizatori}
		\begin{itemize}[noitemsep]
			\item Creare și gestionare conturi utilizatori
			\item Asignare și modificare roluri
			\item Activare/dezactivare conturi
			\item Audit trail pentru toate acțiunile administrative
		\end{itemize}
		
		\item \textbf{Managementul Academic}
		\begin{itemize}[noitemsep]
			\item Definirea și gestionarea planurilor de învățământ
			\item Crearea și asocierea disciplinelor
			\item Înscrirea studenților la materii
			\item Gestionarea promoțiilor și a grupelor de studii
		\end{itemize}
		
		\item \textbf{Generare și Gestionare Orar}
		\begin{itemize}[noitemsep]
			\item Generare automată cu algoritm de optimizare
			\item Editare manuală cu validări în timp real
			\item Gestionarea sălilor și resurselor
			\item Sistem de reprogramări controlat
		\end{itemize}
		
		\item \textbf{Catalog și Notare}
		\begin{itemize}[noitemsep]
			\item Introducere note de către profesori
			\item Calcul automat medii conform regulilor
			\item Istoric academic complet
			\item Raportare restanțe
		\end{itemize}
		
		\item \textbf{Planificare Examene}
		\begin{itemize}[noitemsep]
			\item Definirea sesiunilor de examene
			\item Programare examene cu alocare săli
			\item Înscriere studenți
			\item Gestionarea supraveghetorilor
		\end{itemize}
		
		\item \textbf{Portal Student}
		\begin{itemize}[noitemsep]
			\item Vizualizare orar personal
			\item Acces la catalog și note
			\item Solicitare adeverințe
			\item Consultare plan învățământ
		\end{itemize}
		
		\item \textbf{Portal Profesor}
		\begin{itemize}[noitemsep]
			\item Vizualizare discipline și grupe alocate
			\item Introducere și editare note
			\item Consultare orar personal
			\item Propuneri de reprogramare
		\end{itemize}
		
		\item \textbf{Sistem Notificări}
		\begin{itemize}[noitemsep]
			\item Notificări email
			\item Notificări în platformă (push web)
			\item Alertă pentru evenimente importante
		\end{itemize}
	\end{enumerate}
	
	\subsection{Caracteristici Utilizatori}
	
	Platforma va deservi următoarele categorii de utilizatori:
	
	\begin{table}[H]
		\centering
		\begin{tabular}{|p{3cm}|p{12cm}|}
			\hline
			\rowcolor{tableheader}
			\textbf{Rol} & \textbf{Caracteristici și Nevoi} \\
			\hline
			\textbf{Administrator} & 
			\begin{itemize}[noitemsep,leftmargin=*]
				\item Personal tehnic cu competențe IT avansate
				\item Responsabil pentru configurarea și mentenanța sistemului
				\item Nevoi: Control complet asupra utilizatorilor, rolurilor și configurărilor globale
				\item Acces: CRUD utilizatori, roluri, reset parole, configurări globale
			\end{itemize} \\
			\hline
			\textbf{Secretar} & 
			\begin{itemize}[noitemsep,leftmargin=*]
				\item Personal administrativ cu experiență în procesele academice
				\item Competențe IT medii
				\item Nevoi: Gestionarea eficientă a datelor academice
				\item Acces: CRUD materii, grupe, asocieri profesor–materie–grupă, planuri, programări examene/orar
			\end{itemize} \\
			\hline
			\textbf{Profesor} & 
			\begin{itemize}[noitemsep,leftmargin=*]
				\item Cadre didactice cu competențe IT variabile
				\item Nevoi: Notare rapidă, vizualizare informații studenți, gestionare activități
				\item Acces: Notare, gestionare activități proprii, propuneri reprogramări
			\end{itemize} \\
			\hline
			\textbf{Student} & 
			\begin{itemize}[noitemsep,leftmargin=*]
				\item Vârstă: 19-25 ani, nativi digitali
				\item Competențe IT: Bune (smartphones, aplicații web)
				\item Nevoi: Acces rapid la informații academice, orar, note
				\item Acces: Vizualizare propriile date academice, solicitări adeverințe
			\end{itemize} \\
			\hline
		\end{tabular}
	\end{table}
	
	\subsection{Mediul de Operare}
	
	Platforma va opera în următorul mediu:
	
	\begin{enumerate}
		\item \textbf{Mediu Tehnic}
		\begin{itemize}[noitemsep]
			\item Aplicație web accesibilă prin browsere moderne (Chrome, Firefox, Safari, Edge)
			\item Design responsive pentru acces pe desktop, tablete și smartphones
			\item Hosting: on-premise
		\end{itemize}
		
		\item \textbf{Mediu Hardware}
		\begin{itemize}[noitemsep]
			\item Server(e) cu capacitate de procesare adecvată pentru 300+ utilizatori
			\item Stocare: minimum 500GB pentru baza de date și documente
			\item Backup sistem automat
			\item Conexiune internet stabilă (minimum 100 Mbps)
		\end{itemize}
		
		\item \textbf{Mediu Software}
		\begin{itemize}[noitemsep]
			\item Framework web modern (React + Spring)
			\item Bază de date relațională (PostgreSQL)
			\item Server web (Nginx/Apache)
			\item Sistem de gestiune versiuni (Git)
		\end{itemize}
	\end{enumerate}
	
	% ========== CERINȚE FUNCȚIONALE ==========
	\newpage
	\section{Caracteristici Sistem și Cerințe Funcționale}
	
	Această secțiune detaliază cerințele funcționale ale Platformei de Management Universitar, organizate pe module majore. Fiecare cerință este identificată cu un ID unic (FR-XXX) pentru trasabilitate.
	
	\subsection{Autentificare și Autorizare}
	
	\subsubsection{Autentificare Utilizatori}
	
	\begin{longtable}{|p{2cm}|p{12cm}|}
		\hline
		\rowcolor{tableheader}
		\textbf{ID} & \textbf{Cerință} \\
		\hline
		\endfirsthead
		\hline
		\rowcolor{tableheader}
		\textbf{ID} & \textbf{Cerință} \\
		\hline
		\endhead
		\hline
		\endfoot
		
		FR-001 & Sistemul \reqshall să ofere autentificare prin email și parolă. \\
		\hline
		FR-002 & Sistemul \reqshall să ofere funcționalitate de resetare parolă direct din interfața aplicației. \\
		\hline
		FR-003 & Sistemul \reqshall să ofere opțiunea „Ține-mă minte" care extinde sesiunea la maximum 30 zile. \\
		\hline
	\end{longtable}
	
	\subsubsection{Sistem de Autorizare (RBAC)}
	
	\begin{longtable}{|p{2cm}|p{12cm}|}
		\hline
		\rowcolor{tableheader}
		\textbf{ID} & \textbf{Cerință} \\
		\hline
		FR-004 & Sistemul \reqshall să implementeze RBAC cu următoarele roluri predefinite: Administrator, Secretar, Profesor, Student. \\
		\hline
		FR-005 & Sistemul \reqshall să permită asignarea de roluri multiple unui utilizator (e.g. Profesor + Administrator). \\
		\hline
		FR-006 & Pentru utilizatori cu roluri multiple, sistemul \reqshall să aplice o reuniune a permisiunilor. \\
		\hline
		FR-007 & Sistemul \reqmay să permită administratorilor să definească permisiuni granulare pentru fiecare rol. \\
		\hline
	\end{longtable}
	
	\subsubsection{Permisiuni pe Roluri}
	
	\begin{longtable}{|p{2cm}|p{12cm}|}
		\hline
		\rowcolor{tableheader}
		\textbf{ID} & \textbf{Cerință} \\
		\hline
		FR-008 & Rolul \textbf{Administrator} \reqshall să aibă permisiuni specifice: CRUD utilizatori, CRUD roluri, resetare parole, configurări globale sistem, acces la toate log-urile. \\
		\hline
		FR-009 & Rolul \textbf{Secretar} \reqshall să aibă permisiuni specifice: CRUD materii, CRUD grupe, asociere profesori-materii-grupe, CRUD planuri învățământ, programare examene/orar, aprobare cereri studenți, generare adeverințe. \\
		\hline
		FR-010 & Rolul \textbf{Profesor} \reqshall să aibă permisiuni specifice: citire/modificare note pentru disciplinele proprii, vizualizare liste studenți pentru disciplinele proprii, vizualizare orar propriu, propuneri reprogramări, acces la istoricul academic al studenților din grupele alocate. \\
		\hline
		FR-011 & Rolul \textbf{Student} \reqshall să aibă permisiuni specifice: vizualizare propriile date academice, vizualizare orar grupă, solicitare adeverințe, vizualizare plan învățământ, editare date contact personale (telefon, fotografie). \\
		\hline
	\end{longtable}
	
	\subsection{Management Utilizatori}
	
	\subsubsection{Profil Utilizator}
	
	\begin{longtable}{|p{2cm}|p{12cm}|}
		\hline
		\rowcolor{tableheader}
		\textbf{ID} & \textbf{Cerință} \\
		\hline
		FR-012 & Sistemul \reqshall să stocheze următoarele câmpuri obligatorii pentru fiecare utilizator: nume, prenume, email, telefon, CNP, rol(uri), status (activ/inactiv). \\
		\hline
		FR-013 & Sistemul \reqshall să permită încărcarea unei fotografii de profil (max 2 MB, formate: jpg, png). \\
		\hline
		FR-014 & Sistemul \reqshall să permită studenților și profesorilor să editeze numărul de telefon și fotografia de profil. \\
		\hline
		FR-015 & Sistemul \reqshall să restricționeze editarea emailului și numelui în afara Administratorului. \\
		\hline
		FR-016 & Pentru studenți, sistemul \reqshall să stocheze suplimentar: an studiu, grupă, stare (activ/absolvent/exmatriculat). \\
		\hline
		FR-017 & Pentru profesori, sistemul \reqshall să stocheze suplimentar: grad didactic, discipline alocate. \\
		\hline
	\end{longtable}
	
	\subsubsection{Administrare Conturi}
	
	\begin{longtable}{|p{2cm}|p{12cm}|}
		\hline
		\rowcolor{tableheader}
		\textbf{ID} & \textbf{Cerință} \\
		\hline
		FR-018 & Sistemul \reqshall să permită doar utilizatorilor cu rol Administrator să creeze conturi noi. \\
		\hline
		FR-019 & Sistemul \reqshall să permită administratorilor să activeze/dezactiveze conturi fără a șterge datele. \\
		\hline
		FR-020 & Dezactivarea unui cont \reqshall să blocheze imediat accesul utilizatorului și să marcheze contul ca „inactiv". \\
		\hline
		FR-021 & Sistemul \reqshall să interzică ștergerea conturilor care au activitate înregistrată (note, cereri etc.). Acestea pot fi doar dezactivate. \\
		\hline
		FR-022 & Sistemul \reqshall să permită resetarea parolei de către Administrator pentru orice utilizator. \\
		\hline
		FR-023 & Sistemul \reqshall să trimită notificare prin email utilizatorului când parola este resetată de către Administrator. \\
		\hline
	\end{longtable}
	
	\subsection{Secretariat - Materii și Gestiune Academică}
	
	\subsubsection{Management Materii/Discipline}
	
	\begin{longtable}{|p{2cm}|p{12cm}|}
		\hline
		\rowcolor{tableheader}
		\textbf{ID} & \textbf{Cerință} \\
		\hline
		FR-024 & Sistemul \reqshall să permită secretarului să creeze discipline cu următoarele câmpuri: cod unic, titlu, număr credite, an studiu, semestru, formă (curs/seminar/laborator/proiect), mod evaluare (notă 1-10 sau ADMIS/RESPINS), obiective, conținut, bibliografie, adăugiri. \\
		\hline
		FR-025 & Sistemul \reqshall să asigure unicitatea numelui disciplinei la nivel de instituție. \\
		\hline
		FR-026 & Sistemul \reqshall să permită asocierea unei discipline cu unul sau mai mulți profesori, specificând tipul activității (curs/seminar/laborator/proiect). \\
		\hline
		FR-027 & Sistemul \reqshall să permită asocierea unei activități (curs/seminar/laborator/proiect) cu una sau mai multe grupe. \\
		\hline
		FR-028 & Sistemul \reqshall să valideze capacitatea sălii la asocierea unei activități cu o grupă. \\
		\hline
		FR-029 & Sistemul \reqshall să suporte următoarele stări pentru o disciplină: activă, arhivată. \\
		\hline
		FR-030 & Disciplinele arhivate \reqshall să rămână vizibile în istoricul academic dar nu pot primi noi înscrieri. \\
		\hline
	\end{longtable}
	
	\subsubsection{Înscriere Studenți la Materii}
	
	\begin{longtable}{|p{2cm}|p{12cm}|}
		\hline
		\rowcolor{tableheader}
		\textbf{ID} & \textbf{Cerință} \\
		\hline
		FR-031 & Sistemul \reqshall să înscrie automat studenții la discipline conform planului de învățământ pentru anul și grupa lor. \\
		\hline
	\end{longtable}
	
	\subsubsection{Plan de Învățământ}
	
	\begin{longtable}{|p{2cm}|p{12cm}|}
		\hline
		\rowcolor{tableheader}
		\textbf{ID} & \textbf{Cerință} \\
		\hline
		FR-032 & Sistemul \reqshall să permită secretarului să creeze planuri de învățământ cu următoarele atribute: an universitar, an studiu (1-4), semestre (1-2). \\
		\hline
		FR-033 & Planul de învățământ \reqshall să conține lista disciplinelor cu numărul de credite și tipul activităților. \\
		\hline
		FR-034 & Sistemul \reqshall să valideze că totalul de credite pe semestru este de 30 și pe an de 60 (conform standardelor Bologna). \\
		\hline
		FR-035 & Sistemul \reqshall să suporte versiuni ale planului de învățământ; schimbările nu afectează retroactiv anii anteriori. \\
		\hline
		FR-036 & Sistemul \reqshall să permită duplicarea unui plan existent pentru crearea rapidă a unui plan nou. \\
		\hline
		FR-037 & Sistemul \reqmay să permită exportul planului de învățământ în format PDF sau/și Excel. \\
		\hline
	\end{longtable}
	
	\subsection{Portal Student}
	
	\subsubsection{Vizualizări Informații Academice}
	
	\begin{longtable}{|p{2cm}|p{12cm}|}
		\hline
		\rowcolor{tableheader}
		\textbf{ID} & \textbf{Cerință} \\
		\hline
		FR-038 & Sistemul \reqshall să afișeze studentului lista disciplinelor la care este înscris în semestrul curent. \\
		\hline
		FR-039 & Sistemul \reqshall să permită filtrarea disciplinelor după semestru și an universitar. \\
		\hline
		FR-040 & Pentru fiecare disciplină, sistemul \reqshall să afișeze toate informațiile disponibile conform fișei disciplinei. \\
		\hline
		FR-041 & Sistemul \reqshall să afișeze orarul grupei studentului pentru săptămâna curentă cu posibilitatea navigării între săptămâni. \\
		\hline
		FR-042 & Sistemul \reqshall să evidențieze vizual orele în curs de desfășurare în orar. \\
		\hline
		FR-043 & Sistemul \reqshall să permită studentului accesul read-only la istoricul notelor din anii anteriori. \\
		\hline
		FR-044 & Sistemul \reqshall să afișeze situația academică curentă: discipline înscrise, note obținute, restanțe, medie generală. \\
		\hline
	\end{longtable}
	
	\subsubsection{Cereri Adeverințe}
	
	\begin{longtable}{|p{2cm}|p{12cm}|}
		\hline
		\rowcolor{tableheader}
		\textbf{ID} & \textbf{Cerință} \\
		\hline
		FR-045 & Sistemul \reqshall să permite studenților să solicite următoarele tipuri de adeverințe: adeverință student, adeverință cu drept de bursă, situație școlară. \\
		\hline
		FR-046 & La crearea cererii, sistemul \reqshall să permite studentului să specifice un motiv opțional. \\
		\hline
		FR-047 & Cererile \reqshall să aibă următoarele statusuri: „în lucru", „aprobată", „respinsă". \\
		\hline
		FR-048 & Sistemul \reqshall să notifice secretariatul automat la fiecare cerere nouă. \\
		\hline
		FR-049 & Termenul standard de procesare \reqshall să fie 3 zile lucrătoare. \\
		\hline
		FR-050 & La aprobare, sistemul \reqshould să genereze automat documentul PDF cu semnătură și stampilă electronică. \\
		\hline
		FR-051 & Sistemul \reqshould să notifice studentul la finalizarea cererii (aprobare/respingere). \\
		\hline
		FR-052 & Studentul \reqshould să poate descărca documentul aprobat din platformă. \\
		\hline
		FR-053 & Sistemul \reqshould să păstreze un istoric complet al cererilor (data solicitare, data procesare, status, document generat). \\
		\hline
	\end{longtable}
	
	\subsection{Portal Profesor}
	
	\subsubsection{Notare Studenți}
	
	\begin{longtable}{|p{2cm}|p{12cm}|}
		\hline
		\rowcolor{tableheader}
		\textbf{ID} & \textbf{Cerință} \\
		\hline
		FR-054 & Sistemul \reqshall să permită profesorului introducerea notelor pentru disciplinele la care este asociat. \\
		\hline
		FR-055 & Sistemul \reqshall să suporte următoarele scale de notare: 1-10 (promovat \textgreater 4.5) sau ADMIS/RESPINS. \\
		\hline
		FR-056 & Scala de notare \reqshall să fie definită în fișa disciplinei și aplicată automat. \\
		\hline
		FR-057 & Termenul de introducere a notei finale \reqshall să fie maximum 7 zile de la data examenului. \\
		\hline
		FR-058 & Sistemul \reqshall să notifice studenții în momentul în care au fost evaluați (au primit nota). \\
		\hline
		FR-059 & Sistemul \reqmay să permită profesorului să introducă note prin import CSV (șablon predefinit). \\
		\hline
		FR-060 & La import CSV, sistemul \reqmay să valideze datele și să raporteze erorile înainte de salvare. \\
		\hline
	\end{longtable}
	
	\subsubsection{Modificare Note}
	
	\begin{longtable}{|p{2cm}|p{12cm}|}
		\hline
		\rowcolor{tableheader}
		\textbf{ID} & \textbf{Cerință} \\
		\hline
		FR-061 & Modificările notelor după publicare \reqshall să necesite introducerea unui motiv obligatoriu de către profesor. \\
		\hline
		FR-062 & Modificările semnificative (diferență \textgreater 1 punct sau schimbare status promovat/nepromovat) \reqshall să necesite confirmare din partea secretariatului. \\
		\hline
		FR-063 & Studentul \reqshould să fie notificat automat la modificarea oricărei note. \\
		\hline
		\hline
	\end{longtable}
	
	\subsubsection{Gestionare Activități}
	
	\begin{longtable}{|p{2cm}|p{12cm}|}
		\hline
		\rowcolor{tableheader}
		\textbf{ID} & \textbf{Cerință} \\
		\hline
		FR-064 & Sistemul \reqshall să afișeze profesorului lista tuturor activităților (curs/seminar/laborator/proiect) la care este alocat. \\
		\hline
		FR-065 & Pentru fiecare activitate, sistemul \reqshall să afișeze: disciplina, grupa (sau grupele, în cazul cursurilor), tipul activității, orarul, sala. \\
		\hline
		FR-066 & Sistemul \reqshall să permită profesorului vizualizarea listei de studenți înscrisi la fiecare activitate. \\
		\hline
		FR-067 & Pentru fiecare student, sistemul \reqshall să afișeze: nume, prenume, grupă, fotografie, date contact. \\
		\hline
		FR-068 & Sistemul \reqshall să permită profesorului să propună reprogramarea unei activități cu specificarea noii date și intervalului orar. \\
		\hline
		FR-069 & La propunerea de reprogramare, sistemul \reqshall să valideze disponibilitatea sălii, profesorului și grupei (sau grupelor, în cazul cursurilor) în noul interval. \\
		\hline
		FR-070 & Propunerile de reprogramare \reqshall să necesite aprobare de la secretariat. \\
		\hline
		FR-071 & Sistemul \reqshall să notifice secretariatul și grupa/grupele afectate la orice reprogramare. \\
		\hline
	\end{longtable}
	
	\subsection{Modul Orar}
	
	\subsubsection{Generare Orar}
	
	\begin{longtable}{|p{2cm}|p{12cm}|}
		\hline
		\rowcolor{tableheader}
		\textbf{ID} & \textbf{Cerință} \\
		\hline
		FR-072 & Sistemul \reqshall să ofere două moduri de generare orar: automat (algoritm optimizare) și manual (drag\&drop). \\
		\hline
		FR-073 & Pentru generarea automată, sistemul \reqshall să primească ca input: liste grupe, activități cu durata, săli cu capacitate, indisponibilități profesori. \\
		\hline
		FR-074 & Durata standard a activităților \reqshall să fie 2 ore (configurabil în fișa disciplinei). \\
		\hline
		FR-075 & Sistemul \reqshall să respecte următoarele constrângeri hard: 
		\begin{itemize}[noitemsep]
			\item Fără suprapuneri pentru aceeași grupă
			\item Fără suprapuneri pentru același profesor
			\item Fără suprapuneri pentru aceeași sală
			\item Respectarea capacității sălii
			\item Activități doar în ferestrele de timp instituționale (e.g. 8:00-20:00, luni-vineri)
		\end{itemize} \\
		\hline
		FR-076 & Sistemul \reqshould să optimizeze următoarele constrângeri soft: 
		\begin{itemize}[noitemsep]
			\item Distribuție echilibrată pe zile
			\item Evitarea „ferestrelor" (pauze) \textgreater 2h pentru studenți
			\item Respectarea preferințelor de interval orar ale profesorilor
			\item Minimizarea schimbărilor de săli pentru aceeași grupă
		\end{itemize} \\
		\hline
		FR-077 & În modul manual, sistemul \reqshall să permită secretarului plasarea activităților prin drag\&drop pe grila orarului. \\
		\hline
		FR-078 & La plasare manuală, sistemul \reqshall să valideze în timp real respectarea constrângerilor hard și să prevină plasările invalide. \\
		\hline
		FR-079 & Sistemul \reqshould să evidențieze vizual conflictele și încălcările constrângerilor soft. \\
		\hline
		FR-080 & Sistemul \reqshall să permită salvarea orarului ca draft (nepublicat) pentru editări ulterioare. \\
		\hline
		FR-081 & Publicarea orarului \reqshall să fie posibilă doar dacă nu există conflicte hard. \\
		\hline
		FR-082 & La publicare, sistemul \reqshall să notifice automat toate grupele și profesorii afectați. \\
		\hline
	\end{longtable}
	
	\subsubsection{Vizualizare Orar}
	
	\begin{longtable}{|p{2cm}|p{12cm}|}
		\hline
		\rowcolor{tableheader}
		\textbf{ID} & \textbf{Cerință} \\
		\hline
		FR-083 & Sistemul \reqshall să afișeze orarului în format săptămânal (grilă luni-vineri, 8:00-20:00). \\
		\hline
		FR-084 & Studentul \reqshall să poată vizualiza doar orarul grupei sale. \\
		\hline
		FR-085 & Profesorul \reqshall să poată vizualiza doar activitățile la care este alocat. \\
		\hline
		FR-086 & Secretarul \reqshall să poată vizualiza orarul oricărei grupe sau profesor. \\
		\hline
		FR-087 & Sistemul \reqshall să permită filtrarea orarului după: grupă, profesor, sală, zi, semestru. \\
		\hline
		FR-088 & Pentru fiecare activitate din orar, sistemul \reqshall să afișeze: disciplina, tipul activității, profesorul, sala, intervalul orar. \\
		\hline
		FR-089 & Sistemul \reqmay permite exportul orarului în format PDF și/sau iCal (pentru import în calendar). \\
		\hline
		FR-090 & Sistemul \reqmay evidenția vizual schimbările recente în orar (ultimele 7 zile). \\
		\hline
	\end{longtable}
	
	\subsubsection{Reprogramări}
	
	\begin{longtable}{|p{2cm}|p{12cm}|}
		\hline
		\rowcolor{tableheader}
		\textbf{ID} & \textbf{Cerință} \\
		\hline
		FR-091 & Sistemul \reqshall să permită profesorului sau secretarului să propună reprogramarea unei activități. \\
		\hline
		FR-092 & La propunere, utilizatorul \reqshall să specifice: noua dată, noul interval orar, sala, motiv (opțional). \\
		\hline
		FR-093 & Sistemul \reqshall să valideze imediat disponibilitatea noului slot (profesor, grupă, sală). \\
		\hline
		FR-094 & Propunerile de reprogramare de la profesori \reqshall să necesite aprobare de la secretariat. \\
		\hline
		FR-095 & Reprogramările făcute de secretariat \reqshall să fie aplicate imediat după validare. \\
		\hline
		FR-096 & Sistemul \reqshall să notifice toate persoanele afectate (studenți din grupă, profesor) la orice reprogramare. \\
		\hline
		FR-097 & Sistemul \reqshould să păstreze istoricul complet al reprogramărilor (data inițială, data nouă, motiv, cine a solicitat, cine a aprobat). \\
		\hline
	\end{longtable}
	
	\subsection{Catalog și Gestiune Note}
	
	\subsubsection{Catalog Electronic}
	
	\begin{longtable}{|p{2cm}|p{12cm}|}
		\hline
		\rowcolor{tableheader}
		\textbf{ID} & \textbf{Cerință} \\
		\hline
		FR-098 & Sistemul \reqshall să păstreze catalogul electronic complet pentru fiecare student: toate disciplinele, toate notele, toate încercările. \\
		\hline
		FR-099 & Pentru fiecare notă, sistemul \reqshall să stocheze: valoarea, data introducerii, profesorul care a introdus, încercarea (1, 2, 3...). \\
		\hline
		FR-100 & Sistemul \reqshall să calculeze și să afișeze automat media generală pe semestru și pe an pentru fiecare student. \\
		\hline
		FR-101 & Sistemul \reqshall să identifice automat disciplinele nepromovate (restanțe) și să le marcheze corespunzător. \\
		\hline
		FR-102 & Pentru discipline cu note sub 4.5 (sau RESPINS), sistemul \reqshall să înregistreze numărul încercării (1, 2, 3). \\
		\hline
		FR-103 & Sistemul \reqshall să permită studenților vizualizarea propriului catalog complet (toate semestrele/anii). \\
		\hline
		FR-104 & Sistemul \reqshall să permită profesorilor vizualizarea catalogului doar pentru studenții din grupele lor. \\
		\hline
		FR-105 & Sistemul \reqshall să permită secretarilor vizualizarea catalogului oricărui student. \\
		\hline
		FR-106 & Sistemul \reqmay genera automat foaia matricolă (transcript) în format PDF. \\
		\hline
	\end{longtable}
	
	\subsubsection{Raportare Restanțe}
	
	\begin{longtable}{|p{2cm}|p{12cm}|}
		\hline
		\rowcolor{tableheader}
		\textbf{ID} & \textbf{Cerință} \\
		\hline
		FR-107 & Sistemul \reqshall să genereze un raport centralizat cu toți studenții care au restanțe, grupat pe: an studiu, grupă, disciplină. \\
		\hline
		FR-108 & Pentru fiecare restanță, raportul \reqshall să afișeze: studentul, disciplina, numărul încercării, data ultimei susțineri, nota obținută. \\
		\hline
		FR-109 & Sistemul \reqshall să calculeze și să afișeze perioada următoare de susținere (sesiunea de restanțe). \\
		\hline
		FR-110 & Sistemul \reqshall să notifice automat studenții cu restanțe cu 2 săptămâni înainte de sesiunea de restanțe. \\
		\hline
		FR-111 & Sistemul \reqmay permite exportul raportului de restanțe în format Excel și PDF. \\
		\hline
	\end{longtable}
	
	\subsection{Planificare Examene}
	
	\subsubsection{Definire Sesiuni Examene}
	
	\begin{longtable}{|p{2cm}|p{12cm}|}
		\hline
		\rowcolor{tableheader}
		\textbf{ID} & \textbf{Cerință} \\
		\hline
		FR-112 & Sistemul \reqshall să permită secretarului să definească sesiuni de examene cu următoarele atribute: nume (e.g. „Sesiune ianuarie 2026"), interval calendaristic (data început - data încheiere), tip (examene/restanțe/colocvii). \\
		\hline
		FR-113 & Pentru fiecare examen, sistemul \reqshall să permită definirea unei durate implicite de examen (e.g. 2h, 3h). \\
		\hline
		FR-114 & Sistemul \reqshall să permită programarea examenelor individuale în cadrul sesiunii. \\
		\hline
		FR-115 & Pentru fiecare examen, sistemul \reqshall să stocheze: disciplina, data și ora, durata, sala, supraveghetori (profesori), observații. \\
		\hline
		FR-116 & Sistemul \reqshall să valideze că examenul se încadrează în intervalul sesiunii. \\
		\hline
		FR-117 & Sistemul \reqshall să valideze disponibilitatea sălii, profesorului și grupei în intervalul programat. \\
		\hline
		FR-118 & Sistemul \reqshall să prevină suprapunerea examenelor pentru aceeași grupă de studenți. \\
		\hline
	\end{longtable}
	
	\subsubsection{Publicare Sesiuni de Examinare}
	
	\begin{longtable}{|p{2cm}|p{12cm}|}
		\hline
		\rowcolor{tableheader}
		\textbf{ID} & \textbf{Cerință} \\
		\hline
		FR-119 & Programarea examenelor \reqshall să fie inițial în status „draft" (nepublicată). \\
		\hline
		FR-120 & Secretarul \reqshall să poată publica programarea după validare (i.e. toate disciplinele care au forma respectivă de examinare - examen/restanță/colocviu - să fie încadrate în sesiunea respectivă). \\
		\hline
		FR-121 & La publicare, sistemul \reqshall să notifice automat toți studenții înscriși la disciplinele respective. \\
		\hline
		FR-122 & Lista de participanți \reqshould să fie disponibilă profesorului cu 24h înainte de examen. \\
		\hline
		FR-123 & Sistemul \reqshould să permită modificarea programării examenelor până cu 7 zile înainte, cu notificare automată. \\
		\hline
		FR-124 & Modificările în ultimele 7 zile \reqmay să necesite aprobare specială și notificare urgentă. \\
		\hline
		FR-125 & Sistemul \reqmay să genereze automat liste de prezență pentru fiecare examen (PDF printabil). \\
		\hline
	\end{longtable}
	
	\subsection{Sistem Notificări}
	
	\subsubsection{Canale de Comunicare}
	
	\begin{longtable}{|p{2cm}|p{12cm}|}
		\hline
		\rowcolor{tableheader}
		\textbf{ID} & \textbf{Cerință} \\
		\hline
		FR-126 & Sistemul \reqshall să suporte următoarele canale de notificare: email și push web (notificări în browser). \\
		\hline
		FR-127 & Notificările critice (resetare parolă, dezactivare cont) \reqshall să fie trimise obligatoriu pe email, indiferent de preferințe. \\
		\hline
		FR-128 & Utilizatorii \reqmay să poată configura preferințele de notificare (ce tipuri de evenimente, ce canale). \\
		\hline
	\end{longtable}
	
	\subsubsection{Evenimente Notificări}
	
	\begin{longtable}{|p{2cm}|p{12cm}|}
		\hline
		\rowcolor{tableheader}
		\textbf{ID} & \textbf{Cerință} \\
		\hline
		FR-129 & Sistemul \reqshould să trimită notificări pentru următoarele evenimente:
		\begin{itemize}[noitemsep]
			\item Resetare parolă
			\item Publicare/actualizare note
			\item Aprobare/respingere cerere adeverință
			\item Schimbări în orar (inclusiv reprogramări)
			\item Programare examene
			\item Apropiere termen (examen în 24h, restanțe etc.)
		\end{itemize} \\
		\hline
		FR-130 & Utilizatorii \reqshall să poată vizualiza notificările în platformă într-o secțiune dedicată, cu marchare citit/necitit. \\
		\hline
		FR-131 & Sistemul \reqmay retrimite automat notificările eșuate de maximum 3 ori cu interval exponențial (1h, 4h, 24h). \\
		\hline
	\end{longtable}
	
	% ========== CERINȚE NON-FUNCȚIONALE ==========
	\newpage
	\section{Cerințe Non-Funcționale}
	
	Această secțiune specifică cerințele non-funcționale ale Platformei de Management Universitar, definind atributele de calitate și constrângerile pe care sistemul trebuie să le satisfacă. Fiecare cerință este identificată cu un ID unic (NFR-XXX, respectiv SR-XXX în cazul cerințelor de securitate) pentru trasabilitate.
	
	\subsection{Cerințe de Performanță}
	
	\subsubsection{Timp de Răspuns}
	
	\begin{longtable}{|p{2cm}|p{12cm}|}
		\hline
		\rowcolor{tableheader}
		\textbf{ID} & \textbf{Cerință} \\
		\hline
		NFR-001 & Sistemul \reqshall să încărce paginile standard în mai puțin de 2 secunde în condiții normale. \\
		\hline
		NFR-002 & Sistemul \reqshall să returneze rezultatele căutărilor în mai puțin de 1 secundă pentru interogări standard. \\
		\hline
		NFR-003 & Operațiile de salvare date (e.g. introducere notă) \reqshall să se finalizeze în maximum 3 secunde. \\
		\hline
		NFR-004 & Generarea rapoartelor complexe (e.g. catalog complet) \reqshould să se finalizeze în maximum 10 secunde. \\
		\hline
	\end{longtable}
	
	\subsubsection{Throughput și Scalabilitate}
	
	\begin{longtable}{|p{2cm}|p{12cm}|}
		\hline
		\rowcolor{tableheader}
		\textbf{ID} & \textbf{Cerință} \\
		\hline
		NFR-005 & Sistemul \reqshould să suporte minimum 500 utilizatori concurenți în operațiuni normale. \\
		\hline
		NFR-006 & Sistemul \reqshould să proceseze minimum 100 de introduceri/actualizări note pe minut. \\
		\hline
		NFR-007 & Sistemul \reqshould să facă față unei baze de utilizatori de minimum 5000 (studenți + profesori + secretari + administratori). \\
		\hline
		NFR-008 & Sistemul \reqshould să suporte minimum 50.000 de înregistrări în catalogul de note fără degradarea performanței. \\
		\hline
		NFR-009 & Arhitectura sistemului \reqshould să permită scalare pe orizontală prin adăugarea de servere. \\
		\hline
	\end{longtable}
	
	\subsection{Cerințe de Securitate}
	
	\subsubsection{Autentificare și Sesiuni}
	
	\begin{longtable}{|p{2cm}|p{12cm}|}
		\hline
		\rowcolor{tableheader}
		\textbf{ID} & \textbf{Cerință} \\
		\hline
		SR-001 & Sistemul \reqshall să impună parole cu minimum 10 caractere, incluzând litere mari, litere mici, cifre și caractere speciale. \\
		\hline
		SR-002 & Sistemul \reqshall să blocheze contul după 5 încercări consecutive eșuate de autentificare pentru 30 minute. \\
		\hline
		SR-003 & Sistemul \reqshall să stocheze parolele folosind algoritmi de hashing moderni. \\
		\hline
		SR-004 & Sistemul \reqshall să genereze token-uri de sesiune JWT cu TTL maximum 60 minute și să le invalideze la resetarea parolei. \\
		\hline
		SR-005 & Sistemul \reqmay să impună 2FA obligatorie pentru rolurile Administrator și Secretar. \\
		\hline
	\end{longtable}
	
	\subsubsection{Control de Acces}
	
	\begin{longtable}{|p{2cm}|p{12cm}|}
		\hline
		\rowcolor{tableheader}
		\textbf{ID} & \textbf{Cerință} \\
		\hline
		SR-006 & Sistemul \reqshall să verifice identitatea solicitantului la fiecare acțiune/cerere și să respingă accesul neautorizat cu mesaj explicit. \\
		\hline
		SR-007 & Sistemul \reqshall să aplice principiul privilegiilor minime pentru fiecare rol (Administrator, Secretar, Profesor, Student). \\
		\hline
	\end{longtable}
	
	\subsubsection{Criptare și Protecția Datelor}
	
	\begin{longtable}{|p{2cm}|p{12cm}|}
		\hline
		\rowcolor{tableheader}
		\textbf{ID} & \textbf{Cerință} \\
		\hline
		SR-008 & Sistemul \reqshall să cripteze toate comunicațiile în tranzit. \\
		\hline
		SR-009 & Sistemul \reqshall să cripteze datele sensibile: CNP, parole, token-uri. \\
		\hline
		SR-010 & Baza de date \reqshall să fie accesibilă exclusiv din layer-ul API prin conexiuni criptate. \\
		\hline
	\end{longtable}
	
	\subsubsection{Protecție Împotriva Atacurilor}
	
	\begin{longtable}{|p{2cm}|p{12cm}|}
		\hline
		\rowcolor{tableheader}
		\textbf{ID} & \textbf{Cerință} \\
		\hline
		SR-011 & Sistemul \reqshall să prevină atacuri SQL Injection prin utilizarea exclusivă a interogărilor parametrizate. \\
		\hline
		SR-012 & Sistemul \reqshall să implementeze rate limiting: 5 încercări/minut pentru autentificare, blocare progresivă după încercări eșuate. \\
		\hline
		SR-013 & Sistemul \reqshould să implementeze protecție XSS prin sanitizare input și output encoding. \\
		\hline
		SR-014 & Sistemul \reqshould să prevină CSRF. \\
		\hline
		SR-015 & Sistemul \reqshould să valideze toate input-urile utilizatorului la nivel de back-end. \\
		\hline
	\end{longtable}
	
	\subsubsection{Audit și Jurnalizare}
	
	\begin{longtable}{|p{2cm}|p{12cm}|}
		\hline
		\rowcolor{tableheader}
		\textbf{ID} & \textbf{Cerință} \\
		\hline
		SR-016 & Sistemul \reqshall să înregistreze printr-un mecanism de jurnalizare toate operațiunile cu caracter administrativ: încercări de autentificare (reușite și eșuate), modificări date utilizatori, modificări note, modificări roluri. \\
		\hline
		SR-017 & Fiecare înregistrare \reqshall să conțină: timestamp, ID utilizator, acțiune, resursa afectată. \\
		\hline
		SR-018 & Sistemul \reqshould să păstreze un istoric complet al notificărilor trimise (timestamp, destinatar, tip, canal, status livrare). \\
		\hline
		SR-019 & Sistemul \reqshall să păstreze un istoric complet pentru toate modificările de note (nota veche, nota nouă, data, motiv, cine a modificat). \\
		\hline
		SR-020 & Sistemul \reqshould să păstreze jurnalul de audit minimum 5 ani. \\
		\hline
	\end{longtable}
	
	\subsection{Cerințe de Accesibilitate}
	
	\subsubsection{Interfață Utilizator}
	
	\begin{longtable}{|p{2cm}|p{12cm}|}
		\hline
		\rowcolor{tableheader}
		\textbf{ID} & \textbf{Cerință} \\
		\hline
		NFR-010 & Sistemul \reqshall să ofere o interfață intuitivă, consistentă pe toate paginile. \\
		\hline
		NFR-011 & Sistemul \reqshall să implementeze design responsive. \\
		\hline
		NFR-012 & Sistemul \reqshall să asigure navigare clară cu maximum 3 click-uri pentru funcționalitățile frecvente. \\
		\hline
		NFR-013 & Sistemul \reqshould să ofere feedback vizual pentru toate acțiunile utilizatorului (loading, succes, eroare). \\
		\hline
		NFR-014 & Sistemul \reqshould să minimizeze numărul de câmpuri obligatorii în formulare pentru reducerea fricțiunii. \\
		\hline
	\end{longtable}
	
	% ========== ANEXE ==========
	\newpage
	\section{Anexă}
	Această anexă conține diagramele UML care ilustrează arhitectura și funcționalitățile Platformei de Management Universitar.
	
	\subsection{Diagrame ale Cazurilor de Utilizare}
	Aceste diagrame prezintă cazurile de utilizare pentru fiecare rol principal în sistem.
	
	\begin{figure}[H]
		\centering
		\includegraphics[width=0.8\textwidth]{usecase/administrator.pdf}
		\caption{Diagrama cazurilor de utilizare pentru Administrator}
		\label{fig:usecase-admin}
	\end{figure}
	
	\begin{figure}[H]
		\centering
		\includegraphics[width=\textwidth,height=0.9\textheight,keepaspectratio]{usecase/secretar.pdf}
		\caption{Diagrama cazurilor de utilizare pentru Secretar}
		\label{fig:usecase-secretar}
	\end{figure}
	
	\begin{figure}[H]
		\centering
		\includegraphics[width=\textwidth]{usecase/profesor.pdf}
		\caption{Diagrama cazurilor de utilizare pentru Profesor}
		\label{fig:usecase-profesor}
	\end{figure}
	
	\begin{figure}[H]
		\centering
		\includegraphics[width=0.8\textwidth]{usecase/student.pdf}
		\caption{Diagrama cazurilor de utilizare pentru Student}
		\label{fig:usecase-student}
	\end{figure}
	
	\begin{figure}[H]
		\centering
		\includegraphics[width=0.6\textwidth]{usecase/system-wide.pdf}
		\caption{Diagrama cazurilor de utilizare Generală a Sistemului (Autentificare și Notificări)}
		\label{fig:usecase-system}
	\end{figure}
	
	\newpage
	\subsection{Diagrame de Secvență}
	Aceste diagrame ilustrează fluxurile de interacțiuni secvențiale pentru procesele cheie.
	
	\begin{figure}[H]
		\centering
		\includegraphics[width=\textwidth]{seq/seq_auth.pdf}
		\caption{Diagrama de secvență pentru Autentificare și Autorizare (DFD Auth)}
		\label{fig:seq-auth}
	\end{figure}
	
	\begin{figure}[H]
		\centering
		\includegraphics[width=\textwidth]{seq/seq_management_utilizatori.pdf}
		\caption{Diagrama de secvență pentru Management Utilizatori}
		\label{fig:seq-mgmt-users}
	\end{figure}
	
	\begin{figure}[H]
		\centering
		\includegraphics[width=\textwidth]{seq/seq_gestiune_academica.pdf}
		\caption{Diagrama de secvență pentru Gestiune Academică}
		\label{fig:seq-acad-mgmt}
	\end{figure}
	
	\begin{figure}[H]
		\centering
		\includegraphics[width=\textwidth]{seq/seq_modul_orar.pdf}
		\caption{Diagrama de secvență pentru Modul Orar}
		\label{fig:seq-timetable}
	\end{figure}
	
	\begin{figure}[H]
		\centering
		\includegraphics[width=\textwidth]{seq/seq_planificare_examene.pdf}
		\caption{Diagrama de secvență pentru Planificare Examene}
		\label{fig:seq-exams}
	\end{figure}
	
	\begin{figure}[H]
		\centering
		\includegraphics[width=\textwidth]{seq/seq_catalog.pdf}
		\caption{Diagrama de secvență pentru Catalog Electronic}
		\label{fig:seq-catalog}
	\end{figure}
	
	\begin{figure}[H]
		\centering
		\includegraphics[width=\textwidth]{seq/seq_portal_profesor.pdf}
		\caption{Diagrama de secvență pentru Portal Profesor}
		\label{fig:seq-prof-portal}
	\end{figure}
	
	\begin{figure}[H]
		\centering
		\includegraphics[width=\textwidth]{seq/seq_portal_student.pdf}
		\caption{Diagrama de secvență pentru Portal Student}
		\label{fig:seq-student-portal}
	\end{figure}
	
	\begin{figure}[H]
		\centering
		\includegraphics[width=\textwidth]{seq/seq_sistem_notificari.pdf}
		\caption{Diagrama de secvență pentru Sistem Notificări}
		\label{fig:seq-notif}
	\end{figure}
	
	% ========== SFÂRȘIT DOCUMENT ==========
\end{document}