\documentclass[12pt,a4paper]{article}

% Pachete necesare
\usepackage[utf8]{inputenc}
\usepackage[romanian]{babel}
\usepackage{graphicx}
\usepackage{svg}
\usepackage{geometry}
\usepackage{fancyhdr}
\usepackage{titlesec}
\usepackage{enumitem}
\usepackage{hyperref}
\usepackage{longtable}
\usepackage{array}
\usepackage{booktabs}
\usepackage{xcolor}
\usepackage{float}
\usepackage{colortbl}
\usepackage{tikz}

% Configurare pagină
\geometry{
	left=2.5cm,
	right=2cm,
	top=2.5cm,
	bottom=2.5cm
}

% Configurare hyperlink
\hypersetup{
	hidelinks,
	pdftitle={SRS - Unelte AI in criptografie},
	pdfauthor={Sd.Sg.Maj. Moldovan Andrei}
}

% Header și footer
\pagestyle{fancy}
\fancyhf{}
\fancyhead[L]{\small SRS - Unelte AI in criptografie}
\fancyhead[R]{\small v1.0 - Ianuarie 2026}
\fancyfoot[C]{\thepage}
\setlength{\headheight}{14pt}

% Definire culori
\definecolor{headcolor}{RGB}{0,0,0}            % Negru pentru titluri
\definecolor{tableheader}{RGB}{220,235,247}    % Albastru deschis pentru tabele

% Formatare titluri secțiuni
\titleformat{\section}
{\normalfont\Large\bfseries\color{headcolor}}{\thesection}{1em}{}

\titleformat{\subsection}
{\normalfont\large\bfseries\color{headcolor}}{\thesubsection}{1em}{}

\titleformat{\subsubsection}
{\normalfont\normalsize\bfseries}{\thesubsubsection}{1em}{}

% Comenzi personalizate pentru cerințe
\newcommand{\reqshall}{\textbf{TREBUIE }}
\newcommand{\reqshould}{\textbf{AR TREBUI }}
\newcommand{\reqmay}{\textbf{POATE }}
\newcommand{\projectname}{Unelte software bazate pe mecanisme de inteligență artificială aplicată în criptografie}
\newcommand{\projecttitle}{Specificația Cerințelor Software pentru platforma software de unelte AI în criptografie}
\graphicspath{{./}{doc/SRS/}}
\svgpath{{diagrame/SVGs/}{doc/SRS/diagrame/SVGs/}}
\setsvg{inkscapelatex=false}

\begin{document}

	% ========== PAGINA DE TITLU ==========
	\begin{titlepage}
		\begin{center}

			\textbf{România} \\
			\textbf{Ministerul Apărării Naționale} \\
			\textbf{Academia Tehnică Militară „Ferdinand I"} \\

			\vspace*{0.3cm}

			\textbf{Facultatea de Sisteme Informatice și Securitate Cibernetică} \\
			Calculatoare și sisteme informatice pentru apărare și securitate națională \\

			\vspace*{1cm}

			% Logo ATM
\includegraphics[width=0.35\textwidth]{logo_atm.png}
			\vspace*{1.5cm}

			\large{\parbox{0.9\textwidth}{\centering\textbf{\projectname}}}\\
			\vspace*{0.2cm}
			\normalsize{\projecttitle} \\

			\vspace*{0.5cm}

			\normalsize
			Disciplina: Ingineria Programării \\

			\vspace*{0.8cm}

			\small
			\begin{tabular*}{\textwidth}{l@{\extracolsep{\fill}}r}
				\textbf{Profesor Coordonator} & \textbf{Student} \\
				Prof. Dr. Ing. Mihai Togan & Sd.Sg.Maj. Moldovan Andrei \\
			\end{tabular*}
			\normalsize

			\vspace*{1.5cm}

			București \\
			4 Ianuarie 2026

		\end{center}
	\end{titlepage}

	% ========== CONTROL DOCUMENT ==========
	\newpage
	\section*{Control Document}

	\begin{table}[H]
		\centering
		\begin{tabular}{|p{4cm}|p{10cm}|}
			\hline
			\textbf{Titlu} & \projecttitle \\
			\hline
			\textbf{Data} & 4 Ianuarie 2026 \\
			\hline
			\textbf{Status} & Draft \\
			\hline
			\textbf{Versiune} & 1.0 \\
			\hline
			\textbf{Pregătit pentru} & Academia Tehnică Militară „Ferdinand I" \\
			\hline
			\textbf{Referință} & SRS\_AI\_CRYPTO\_V1.0\_Jan\_2026 \\
			\hline
		\end{tabular}
	\end{table}

	\subsection*{Disclaimer}
	Acest document este pregătit în scopuri academice pentru proiectul de curs la disciplina Ingineria Programării. Documentul conține specificațiile detaliate ale cerințelor software pentru \projectname.

	% ========== CUPRINS ==========
	\newpage
	\tableofcontents

	% ========== GLOSAR ==========
	\newpage
	\section*{Glosar}
	\addcontentsline{toc}{section}{Glosar}

	\begin{table}[H]
		\centering
		\begin{tabular}{|p{3cm}|p{11cm}|}
			\hline
			\rowcolor{tableheader}
			\textbf{Termen} & \textbf{Definiție} \\
			\hline
			API & Application Programming Interface \\
			\hline
			RAG & Retrieval-Augmented Generation \\
			\hline
			RBAC & Role-Based Access Control \\
			\hline
			SRS & Software Requirements Specification \\
			\hline
			JWT & JSON Web Token \\
			\hline
			MFA & Multi-Factor Authentication \\
			\hline
			TLS & Transport Layer Security \\
			\hline
			mTLS & Mutual TLS \\
			\hline
			RL & Reinforcement Learning \\
			\hline
			LLM & Large Language Model \\
			\hline
			ML & Machine Learning \\
			\hline
			PQC & Post-Quantum Cryptography \\
			\hline
			HIBP & Have I Been Pwned \\
			\hline
			YAFU & Yet Another Factorization Utility \\
			\hline
			CSP & Content Security Policy \\
			\hline
			CSRF & Cross-Site Request Forgery \\
			\hline
			XSS & Cross-Site Scripting \\
			\hline
			TTL & Time to Live \\
			\hline
			K8s & Kubernetes \\
			\hline
			Docker & Containerization Platform \\
			\hline
			GDPR & General Data Protection Regulation \\
			\hline
			SIEM & Security Information and Event Management \\
			\hline
		\end{tabular}
	\end{table}

	% ========== INTRODUCERE ==========
	\newpage
	\section{Introducere}

	\subsection{Scopul Documentului}

	Acest document SRS oferă o descriere completă a platformei software de unelte AI în criptografie, un sistem de inteligență criptografică de următoarea generație, bazat pe o arhitectură multi-agent autonomă. Platforma orchestrează agenți AI specializați pentru analiza securității parolelor, factorizare numere prime, asistență teoretică în criptografie, operațiuni criptografice și detecție de criptosisteme cu scor de încredere.

	Platforma facilitează:
	\begin{itemize}[noitemsep]
		\item Evaluarea securității parolelor folosind ansambluri ML
		\item Verificarea primalității și factorizarea numerelor mari
		\item Asistență teoretică în criptografie prin RAG
		\item Execuția de operațiuni criptografice (simetrice, asimetrice, PQC)
		\item Detectarea automată a criptosistemelor din ciphertext
	\end{itemize}

	Acest SRS va servi drept bază pentru fazele ulterioare de proiectare, dezvoltare și testare ale sistemului.

	\subsection{Convenții Document}

	Acest document urmează convențiile RFC 2119 cu următoarele traduceri în limba română:

	\begin{table}[H]
		\centering
		\begin{tabular}{|p{3cm}|p{11cm}|}
			\hline
			\rowcolor{tableheader}
			\textbf{Termen} & \textbf{Descriere} \\
			\hline
			\reqshall & Indică o cerință obligatorie care trebuie implementată. \\
			\hline
			\reqshould & Indică o cerință recomandată. \\
			\hline
			\reqmay & Indică o cerință opțională. \\
			\hline
		\end{tabular}
	\end{table}

	Cerințele sunt categorisite astfel:

	\begin{table}[H]
		\centering
		\begin{tabular}{|p{4cm}|p{10cm}|}
			\hline
			\rowcolor{tableheader}
			\textbf{Număr Cerință} & \textbf{Descriere} \\
			\hline
			FR-XXX-NNN & Cerințe Funcționale \\
			\hline
			NFR-XXX-NNN & Cerințe Non-Funcționale \\
			\hline
			SR-XXX-NNN & Cerințe de Securitate \\
			\hline
			ML-XXX-NNN & Cerințe AI/ML \\
			\hline
			INF-XXX-NNN & Cerințe de Infrastructură \\
			\hline
			CMP-XXX-NNN & Cerințe de Conformitate \\
			\hline
		\end{tabular}
	\end{table}

	\subsection{Cadrul Proiectului}

	\subsubsection{În Cadrul Proiectului}

	Platforma software de unelte AI în criptografie include următoarele componente principale:

	\begin{enumerate}[noitemsep]
		\item Modul de Identity și Access Management (IAM)
		\item Orchestrator Multi-Agent
		\item Agenți Specializați (6 agenți)
		\item Sistem Notificări și Conversații
		\item Interfață Web React și CLI
	\end{enumerate}

	\subsubsection{În Afara Domeniului Proiectului}

	Următoarele aspecte sunt excluse explicit din acest proiect:

	\begin{enumerate}[noitemsep]
		\item Sistem de plăți sau monetizare
		\item Aplicație mobilă nativă (iOS/Android)
		\item Integrare cu sisteme externe de învățământ
		\item Sistem de management al utilizatorilor la scară enterprise
		\item Suport pentru mai multe limbi (doar engleză)
	\end{enumerate}

	% ========== DESCRIERE GENERALĂ ==========
	\newpage
	\section{Descriere Generală}

	\subsection{Descrierea produsului}

	Platforma este o soluție integrată de unelte AI aplicate în criptografie, complet funcțională, care unifică agenți specializați pentru criptanaliză, audit de parole și suport educațional. Sistemul include un orchestrator central pentru rutare, scheduling și execuție asincronă a job-urilor; un agent de decizie pentru extragerea intenției și a entităților; un detector de criptosisteme care are la baza RL cu scor de încredere; un modul Hash Breaker cu integrare HashCat/John și generare controlată de parole; un agregator de scoruri de robustețe (ML, zxcvbn, verificări de tip breach); un serviciu de primalitate/factorizare (YAFU + FactorDB); un modul RAG local pentru asistență teoretică; și un CTF Tool dedicat scenariilor educaționale și testare reproductibilă. Arhitectura este bazată pe microservicii containerizate, orchestrate în Kubernetes, expune API/CLI/Web, include autentificare și autorizare RBAC, audit și monitorizare continuă, iar pipeline-ul CI/CD și publicarea open-source pe GitHub susțin livrarea production-ready atât local, cât și în cloud.

	\subsection{Perspectiva Produsului}

Platforma software de unelte AI în criptografie este dezvoltată ca proiect de licență la Academia Tehnică Militară „Ferdinand I". Sistemul operează ca o aplicație web centralizată, accesibilă prin browsere standard și CLI, fără a necesita instalarea de software client.

	\subsection{Funcționalitățile Produsului}

Platforma software de unelte AI în criptografie oferă următoarele funcționalități majore:

	\begin{enumerate}
		\item \textbf{Autentificare și Managementul Accesului}
		\begin{itemize}[noitemsep]
			\item Autentificare securizată cu email și parolă
			\item Sistem RBAC cu 3 roluri (Anonymous, User, Admin)
			\item Suport MFA pentru admini
			\item Token-uri JWT cu rotație automată
		\end{itemize}

		\item \textbf{Orchestrare Multi-Agent}
		\begin{itemize}[noitemsep]
			\item Detectare automată a intenției utilizatorului
			\item Routing inteligent către agenții potriviți
			\item Execuție paralelă pentru operații independente
			\item Agregare răspunsuri multiple
		\end{itemize}

		\item \textbf{Verificare Parole (Password Checker)}
		\begin{itemize}[noitemsep]
			\item Evaluare ML cu PassGPT, zxcvbn, PassStrengthAI
			\item Verificare HIBP (k-anonymity)
			\item Scor unificat 0-100
			\item Recomandări acționabile
		\end{itemize}

		\item \textbf{Verificare Primalitate (Prime Checker)}
		\begin{itemize}[noitemsep]
			\item Test Miller-Rabin deterministic
			\item Factorizare cu YAFU și FactorDB
			\item Cache LRU în memorie și persistent
		\end{itemize}

		\item \textbf{Specialist Teorie (Theory Specialist)}
		\begin{itemize}[noitemsep]
			\item RAG pentru criptografie cu ChromaDB
			\item Ingestie PDF, Markdown, Text
			\item Reranking cu cross-encoder
			\item Istoric conversații cu context
		\end{itemize}

		\item \textbf{Executor Comenzi (Command Executor)}
		\begin{itemize}[noitemsep]
			\item Operații crypto: AES, RSA, HMAC, PQC
			\item Hashing: SHA-256/384/512, SHA3, BLAKE2
			\item Encoding: Base64, Hex
			\item Implementare în Rust pentru siguranță
		\end{itemize}

		\item \textbf{Selector Alegeri (Choice Maker)}
		\begin{itemize}[noitemsep]
			\item NLP pentru clasificare intenții
			\item Extracție entități cu SecureBERT 2.0
			\item 10+ clase de intenție
		\end{itemize}

		\item \textbf{Detectare Criptosisteme}
		\begin{itemize}[noitemsep]
			\item Integrare CyberChef Magic
			\item Euristici dcode-like
			\item Scor de încredere 0-1
		\end{itemize}


	\end{enumerate}

	\subsection{Caracteristici Utilizatori}

	Platforma deservește următoarele categorii de utilizatori:

	\begin{table}[H]
		\centering
		\begin{tabular}{|p{3cm}|p{12cm}|}
			\hline
			\rowcolor{tableheader}
			\textbf{Rol} & \textbf{Caracteristici și Nevoi} \\
			\hline
			\textbf{Anonymous} &
			\begin{itemize}[noitemsep,leftmargin=*]
				\item Utilizatori neautentificați
				\item Acces limitat la operații demo
				\item Nevoi: Testare funcționalități de bază
			\end{itemize} \\
			\hline
			\textbf{User} &
			\begin{itemize}[noitemsep,leftmargin=*]
				\item Utilizatori înregistrați
				\item Competențe IT: medii-avansate
				\item Nevoi: Acces complet la toate agenții, istoric conversații
			\end{itemize} \\
			\hline
			\textbf{Admin} &
			\begin{itemize}[noitemsep,leftmargin=*]
				\item Administratori sistem
				\item Competențe IT: avansate
				\item Nevoi: Management utilizatori, configurări globale, audit, monitorizare
			\end{itemize} \\
			\hline
		\end{tabular}
	\end{table}

	\subsection{Mediul de Operare}

	Platforma va opera în următorul mediu:

	\begin{enumerate}
		\item \textbf{Mediu Tehnic}
		\begin{itemize}[noitemsep]
			\item Aplicație web accesibilă prin browsere moderne
			\item CLI pentru power users
			\item Design responsive
			\item Hosting: on-premise sau cloud (AWS/GCP/Azure)
		\end{itemize}

		\item \textbf{Mediu Hardware}
		\begin{itemize}[noitemsep]
			\item Server(e) cu capacitate de procesare adecvată pentru 100+ utilizatori
			\item Stocare: minimum 100GB pentru baze de date și modele ML
			\item Backup automat
		\end{itemize}

		\item \textbf{Mediu Software}
		\begin{itemize}[noitemsep]
			\item Containerizare: Docker + Docker Compose
			\item Orchestrare: Kubernetes (producție)
			\item Limbaje: Go, Rust, Python, TypeScript
			\item Baze de date: PostgreSQL 16, Redis 7, ChromaDB, BoltDB
			\item Frontend: React
			\item Observabilitate: Prometheus + Grafana
		\end{itemize}
	\end{enumerate}

	% ========== CERINȚE FUNCȚIONALE ==========
	\newpage
	\section{Cerințe Funcționale}

Această secțiune detaliază cerințele funcționale ale platformei software de unelte AI în criptografie, organizate pe module majore.

	\subsection{Identity și Access Management (IAM)}

	\subsubsection{Autentificare și Autorizare}

	\begin{longtable}{|p{2cm}|p{12cm}|}
		\hline
		\rowcolor{tableheader}
		\textbf{ID} & \textbf{Cerință} \\
		\hline
		\endfirsthead
		\hline
		\rowcolor{tableheader}
		\textbf{ID} & \textbf{Cerință} \\
		\hline
		\endhead
		\hline
		\endfoot

		FR-IAM-001 & Sistemul \reqshall să permită înregistrarea utilizatorilor cu email, parolă și confirmare email. \\
		\hline
		FR-IAM-002 & Sistemul \reqshall să implementeze autentificare prin email/parolă cu rate limiting (max 5 încercări/minut). \\
		\hline
		FR-IAM-003 & Sistemul \reqshall să suporte MFA (TOTP RFC 6238) pentru rolul Admin. \\
		\hline
		FR-IAM-004 & Sistemul \reqshould să suporte WebAuthn/FIDO2 pentru passwordless authentication. \\
		\hline
		FR-IAM-005 & Sistemul \reqshall să emită token-uri JWT (access: 15min, refresh: 7 zile) cu rotație automată. \\
		\hline
		FR-IAM-006 & Sistemul \reqshall să permită generarea și revocarea de API keys cu scope-uri configurabile. \\
		\hline
		FR-IAM-007 & Sistemul \reqshall să implementeze RBAC cu 3 roluri predefinite: Anonymous, User, Admin. \\
		\hline
		FR-IAM-008 & Sistemul \reqshall să permită resetarea parolei prin email cu token unic (TTL: 1 oră). \\
		\hline
		FR-IAM-009 & Sistemul \reqshould să suporte OAuth2/OIDC pentru autentificare externă (GitHub, Google). \\
		\hline
		FR-IAM-010 & Sistemul \reqshall să invalideze toate sesiunile active la schimbarea parolei. \\
		\hline
	\end{longtable}

	\subsection{Orchestrare Multi-Agent}

	\subsubsection{Orchestrator}

	\begin{longtable}{|p{2cm}|p{12cm}|}
		\hline
		\rowcolor{tableheader}
		\textbf{ID} & \textbf{Cerință} \\
		\hline
		\endfirsthead
		\hline
		\rowcolor{tableheader}
		\textbf{ID} & \textbf{Cerință} \\
		\hline
		\endhead
		\hline
		\endfoot

		FR-ORC-001 & Orchestratorul \reqshall să detecteze intenția utilizatorului folosind agentul Choice Maker. \\
		\hline
		FR-ORC-002 & Orchestratorul \reqshall să ruteze cererile către agentul/agenții potriviți pe baza intenției detectate. \\
		\hline
		FR-ORC-003 & Orchestratorul \reqshall să suporte execuție paralelă pentru operații independente. \\
		\hline
		FR-ORC-004 & Orchestratorul \reqshall să agregeze răspunsurile de la mai mulți agenți într-un răspuns unificat. \\
		\hline
		FR-ORC-005 & Orchestratorul \reqshall să implementeze timeout configurabil per agent (default: 30s). \\
		\hline
		FR-ORC-006 & Orchestratorul \reqshall să implementeze circuit breaker pentru agenți cu probleme. \\
		\hline
		FR-ORC-007 & Orchestratorul \reqshall să expună health endpoints pentru fiecare serviciu gestionat. \\
		\hline
		FR-ORC-008 & Orchestratorul \reqshould să ofere fallback logic când agenții sunt indisponibili. \\
		\hline
		FR-ORC-009 & Orchestratorul \reqshall să suporte selectarea dinamică a provider-ului LLM per cerere. \\
		\hline
		FR-ORC-010 & Orchestratorul \reqshould să permită configurarea priorităților de rutare per agent. \\
		\hline
	\end{longtable}

	\subsection{Agent Verificare Parole}

	\subsubsection{Password Intelligence}

	\begin{longtable}{|p{2cm}|p{12cm}|}
		\hline
		\rowcolor{tableheader}
		\textbf{ID} & \textbf{Cerință} \\
		\hline
		\endfirsthead
		\hline
		\rowcolor{tableheader}
		\textbf{ID} & \textbf{Cerință} \\
		\hline
		\endhead
		\hline
		\endfoot

		FR-PWD-001 & Agentul \reqshall să calculeze scorul de securitate unificat (0-100) din ansamblu ML. \\
		\hline
		FR-PWD-002 & Agentul \reqshall să integreze PassGPT pentru analiza probabilistică a parolelor. \\
		\hline
		FR-PWD-003 & Agentul \reqshall să integreze zxcvbn pentru evaluarea heuristică. \\
		\hline
		FR-PWD-004 & Agentul \reqshall să verifice parola contra bazei HIBP (k-anonymity). \\
		\hline
		FR-PWD-005 & Agentul \reqshould să integreze PassStrengthAI (CNN) pentru evaluare suplimentară. \\
		\hline
		FR-PWD-006 & Agentul \reqshall să returneze recomandări acționabile pentru îmbunătățirea parolei. \\
		\hline
		FR-PWD-007 & Agentul \reqshall să dezactiveze automat PassGPT pentru parole > 10 caractere. \\
		\hline
		FR-PWD-008 & Agentul \reqshould să aplice penalizări pentru parole scurte (< 8 caractere). \\
		\hline
		FR-PWD-009 & Agentul \reqshall să limiteze lungimea parolei acceptate la 128 caractere. \\
		\hline
		FR-PWD-010 & Agentul NU \reqshall să stocheze sau să logheze parola în clar. \\
		\hline
	\end{longtable}

	\subsection{Agent Verificare Primalitate}

	\subsubsection{Prime Factorization}

	\begin{longtable}{|p{2cm}|p{12cm}|}
		\hline
		\rowcolor{tableheader}
		\textbf{ID} & \textbf{Cerință} \\
		\hline
		\endfirsthead
		\hline
		\rowcolor{tableheader}
		\textbf{ID} & \textbf{Cerință} \\
		\hline
		\endhead
		\hline
		\endfoot

		FR-PRM-001 & Agentul \reqshall să verifice primalitatea numerelor folosind Miller-Rabin deterministic pentru numere < 2$^{\wedge}$64. \\
		\hline
		FR-PRM-002 & Agentul \reqshall să integreze YAFU pentru factorizare avansată. \\
		\hline
		FR-PRM-003 & Agentul \reqshall să utilizeze FactorDB ca fallback pentru numere mari. \\
		\hline
		FR-PRM-004 & Agentul \reqshall să implementeze cache LRU in-memory + persistent BoltDB. \\
		\hline
		FR-PRM-005 & Agentul \reqshall să returneze factorii primi și metoda folosită. \\
		\hline
		FR-PRM-006 & Agentul \reqshall să limiteze numărul maxim de cifre acceptate (default: 1000). \\
		\hline
		FR-PRM-007 & Agentul \reqshall să implementeze timeout-uri per backend (YAFU: 5s primality, 8s factor). \\
		\hline
		FR-PRM-008 & Agentul \reqshould să raporteze timpul de calcul în răspuns. \\
		\hline
		FR-PRM-009 & Agentul \reqshall să expună endpoint /history pentru ultimele rezultate. \\
		\hline
		FR-PRM-010 & Agentul \reqshall să gestioneze concurența YAFU cu semaphore (default: 2). \\
		\hline
	\end{longtable}

	\subsection{Agent Specialist Teorie}

	\subsubsection{RAG pentru Criptografie}

	\begin{longtable}{|p{2cm}|p{12cm}|}
		\hline
		\rowcolor{tableheader}
		\textbf{ID} & \textbf{Cerință} \\
		\hline
		\endfirsthead
		\hline
		\rowcolor{tableheader}
		\textbf{ID} & \textbf{Cerință} \\
		\hline
		\endhead
		\hline
		\endfoot

		FR-RAG-001 & Agentul \reqshall să suporte ingestia documentelor PDF, Markdown și Text. \\
		\hline
		FR-RAG-002 & Agentul \reqshall să stocheze embeddings în ChromaDB cu persistență. \\
		\hline
		FR-RAG-003 & Agentul \reqshall să utilizeze FastEmbed (BAAI/bge-small-en-v1.5) pentru vectorizare. \\
		\hline
		FR-RAG-004 & Agentul \reqshall să implementeze reranking cu cross-encoder (BAAI/bge-reranker-base). \\
		\hline
		FR-RAG-005 & Agentul \reqshall să mențină istoricul conversațiilor cu context tracking. \\
		\hline
		FR-RAG-006 & Agentul \reqshall să returneze surse (citări) pentru fiecare răspuns generat. \\
		\hline
		FR-RAG-007 & Agentul \reqshould să suporte hybrid retrieval (vector + BM25). \\
		\hline
		FR-RAG-008 & Agentul \reqshall să suporte multiple LLM providers (Ollama, OpenAI, Gemini). \\
		\hline
		FR-RAG-009 & Agentul \reqshall să permită auto-ingestia documentelor noi din folder monitorizat. \\
		\hline
		FR-RAG-010 & Agentul \reqshould să permită selectarea direct\_rag pentru bypass LLM. \\
		\hline
	\end{longtable}

	\subsection{Agent Executor Comenzi}

	\subsubsection{Operații Criptografice}

	\begin{longtable}{|p{2cm}|p{12cm}|}
		\hline
		\rowcolor{tableheader}
		\textbf{ID} & \textbf{Cerință} \\
		\hline
		\endfirsthead
		\hline
		\rowcolor{tableheader}
		\textbf{ID} & \textbf{Cerință} \\
		\hline
		\endhead
		\hline
		\endfoot

		FR-CMD-001 & Agentul \reqshall să suporte operații de encoding: Base64, Hex. \\
		\hline
		FR-CMD-002 & Agentul \reqshall să suporte hashing: SHA-256/384/512, SHA3, BLAKE2, MD5, HMAC. \\
		\hline
		FR-CMD-003 & Agentul \reqshall să suporte criptare simetrică AES-CBC + HMAC (Encrypt-then-MAC). \\
		\hline
		FR-CMD-004 & Agentul \reqshall să suporte criptare asimetrică RSA cu OAEP padding. \\
		\hline
		FR-CMD-005 & Agentul \reqshall să suporte semnături post-quantum (ML-DSA/Dilithium, Falcon). \\
		\hline
		FR-CMD-006 & Agentul \reqshall să valideze toate inputurile contra injection attacks. \\
		\hline
		FR-CMD-007 & Agentul \reqshall să redacteze secretele din logs/erori. \\
		\hline
		FR-CMD-008 & Agentul \reqshall să returneze comanda OpenSSL executată (scop educațional). \\
		\hline
		FR-CMD-009 & Agentul \reqshall să implementeze timeout per operație (default: 30s). \\
		\hline
		FR-CMD-010 & Agentul \reqshall să raporteze disponibilitatea PQC provider la /pqc/health. \\
		\hline
	\end{longtable}

	\subsection{Agent Selector Alegeri}

	\subsubsection{NLP Intent Classification}

	\begin{longtable}{|p{2cm}|p{12cm}|}
		\hline
		\rowcolor{tableheader}
		\textbf{ID} & \textbf{Cerință} \\
		\hline
		\endfirsthead
		\hline
		\rowcolor{tableheader}
		\textbf{ID} & \textbf{Cerință} \\
		\hline
		\endhead
		\hline
		\endfoot

		FR-NLP-001 & Agentul \reqshall să clasifice intenția utilizatorului cu confidence score. \\
		\hline
		FR-NLP-002 & Agentul \reqshall să extragă entități relevante (numere, algoritmi, parole, chei). \\
		\hline
		FR-NLP-003 & Agentul \reqshall să utilizeze SecureBERT 2.0 pentru clasificare. \\
		\hline
		FR-NLP-004 & Agentul \reqshall să suporte minim 10 clase de intenție (encrypt, decrypt, hash, etc.). \\
		\hline
		FR-NLP-005 & Agentul \reqshall să returneze threshold de confidence configurabil. \\
		\hline
		FR-NLP-006 & Agentul \reqshould să detecteze cereri ambigue și să solicite clarificare. \\
		\hline
		FR-NLP-007 & Agentul \reqshall să proceseze cereri în limba engleză. \\
		\hline
		FR-NLP-008 & Agentul \reqmay să suporte input multilingv cu traducere automată. \\
		\hline
	\end{longtable}

	\subsection{Agent Detectare Criptosisteme}

	\subsubsection{Cryptosystem Detection}

	\begin{longtable}{|p{2cm}|p{12cm}|}
		\hline
		\rowcolor{tableheader}
		\textbf{ID} & \textbf{Cerință} \\
		\hline
		\endfirsthead
		\hline
		\rowcolor{tableheader}
		\textbf{ID} & \textbf{Cerință} \\
		\hline
		\endhead
		\hline
		\endfoot

		FR-CRY-001 & Agentul \reqshall să detecteze tipul de criptosistem din ciphertext. \\
		\hline
		FR-CRY-002 & Agentul \reqshall să integreze CyberChef Magic detector. \\
		\hline
		FR-CRY-003 & Agentul \reqshould să integreze euristici inspirate din dcode.fr. \\
		\hline
		FR-CRY-004 & Agentul \reqshall să agregeze rezultatele de la mai mulți detectori. \\
		\hline
		FR-CRY-005 & Agentul \reqshall să returneze scor de incredere (0-1) pentru fiecare detecție. \\
		\hline
		FR-CRY-006 & Agentul \reqshall să returneze top N candidați ordonați după scor (N configurabil). \\
		\hline
	\end{longtable}

	\subsection{Management Date și Conversații}

	\subsubsection{Data Persistence}

	\begin{longtable}{|p{2cm}|p{12cm}|}
		\hline
		\rowcolor{tableheader}
		\textbf{ID} & \textbf{Cerință} \\
		\hline
		\endfirsthead
		\hline
		\rowcolor{tableheader}
		\textbf{ID} & \textbf{Cerință} \\
		\hline
		\endhead
		\hline
		\endfoot

		FR-DAT-001 & Sistemul \reqshall să stocheze istoricul conversațiilor per utilizator. \\
		\hline
		FR-DAT-002 & Sistemul \reqshall să permită reluarea conversațiilor anterioare. \\
		\hline
		FR-DAT-003 & Sistemul \reqshall să permită exportul rezultatelor în JSON. \\
		\hline
		FR-DAT-004 & Sistemul \reqshould să permită exportul rapoartelor în PDF. \\
		\hline
		FR-DAT-005 & Sistemul \reqshall să implementeze TTL configurabil pentru cache (default: 1h). \\
		\hline
		FR-DAT-006 & Sistemul \reqshall să permită ștergerea datelor utilizatorului la cerere (GDPR). \\
		\hline
		FR-DAT-007 & Sistemul \reqshall să anonimizeze datele în log-uri. \\
		\hline
		FR-DAT-008 & Sistemul \reqshould să implementeze backup automat al bazelor de date. \\
		\hline
		FR-DAT-009 & Sistemul \reqshall să definească retention policy pentru date (default: 90 zile). \\
		\hline
		FR-DAT-010 & Sistemul \reqshould să permită exportul metadatelor conversațiilor. \\
		\hline
	\end{longtable}

	\subsection{Administrare și Audit}

	\subsubsection{System Administration}

	\begin{longtable}{|p{2cm}|p{12cm}|}
		\hline
		\rowcolor{tableheader}
		\textbf{ID} & \textbf{Cerință} \\
		\hline
		\endfirsthead
		\hline
		\rowcolor{tableheader}
		\textbf{ID} & \textbf{Cerință} \\
		\hline
		\endhead
		\hline
		\endfoot

		FR-ADM-001 & Sistemul \reqshall să logheze toate acțiunile administrative în audit log. \\
		\hline
		FR-ADM-002 & Sistemul \reqshall să înregistreze timestamp, user ID, acțiune, resursa afectată, IP. \\
		\hline
		FR-ADM-003 & Sistemul \reqshall să ofere UI de administrare pentru utilizatori și roluri. \\
		\hline
		FR-ADM-004 & Sistemul \reqshall să ofere dashboard pentru management API keys. \\
		\hline
		FR-ADM-005 & Sistemul \reqshall să implementeze rate limiting configurabil per endpoint. \\
		\hline
		FR-ADM-006 & Sistemul \reqshall să implementeze quota per utilizator/API key. \\
		\hline
		FR-ADM-007 & Sistemul \reqshould să alerteze la pattern-uri anormale (brute force, anomalii). \\
		\hline
		FR-ADM-008 & Sistemul \reqshall să permită configurări centralizate per mediu (dev/staging/prod). \\
		\hline
		FR-ADM-009 & Sistemul \reqshall să păstreze audit log-ul minim 5 ani. \\
		\hline
		FR-ADM-010 & Sistemul \reqshould să ofere export audit log în format SIEM-compatible. \\
		\hline
	\end{longtable}

	\subsection{Interfață Utilizator}

	\subsubsection{UI și UX}

	\begin{longtable}{|p{2cm}|p{12cm}|}
		\hline
		\rowcolor{tableheader}
		\textbf{ID} & \textbf{Cerință} \\
		\hline
		\endfirsthead
		\hline
		\rowcolor{tableheader}
		\textbf{ID} & \textbf{Cerință} \\
		\hline
		\endhead
		\hline
		\endfoot

		FR-UI-001 & Interfața web \reqshall să ofere input conversațional pentru cereri. \\
		\hline
		FR-UI-002 & Interfața \reqshall să afișeze rezultatele într-un format structurat și lizibil. \\
		\hline
		FR-UI-003 & Interfața \reqshall să afișeze sursele (citări) pentru răspunsurile RAG. \\
		\hline
		FR-UI-004 & Interfața \reqshall să permită navigarea între conversații anterioare. \\
		\hline
		FR-UI-005 & Interfața \reqshall să fie responsive pentru desktop, tabletă și mobil. \\
		\hline
		FR-UI-006 & Interfața \reqshould să ofere mod întunecat (dark mode). \\
		\hline
		FR-UI-007 & Interfața \reqshall să afișeze status de loading pentru operații async. \\
		\hline
		FR-UI-008 & Interfața \reqshall să afișeze erori într-un mod user-friendly. \\
		\hline
		FR-UI-009 & CLI-ul \reqshall să ofere acces la toate funcționalitățile core. \\
		\hline
		FR-UI-010 & CLI-ul \reqshould să suporte output în format JSON pentru scripting. \\
		\hline
	\end{longtable}

	% ========== CERINȚE NON-FUNCȚIONALE ==========
	\newpage
	\section{Cerințe Non-Funcționale}

	\subsection{Performanță}

	\subsubsection{Timp de Răspuns}

	\begin{longtable}{|p{2cm}|p{12cm}|}
		\hline
		\rowcolor{tableheader}
		\textbf{ID} & \textbf{Cerință} \\
		\hline
		\endfirsthead
		\hline
		\rowcolor{tableheader}
		\textbf{ID} & \textbf{Cerință} \\
		\hline
		\endhead
		\hline
		\endfoot

		NFR-PRF-001 & Endpoint-urile lightweight (health, status) \reqshall să răspundă în p95 < 100ms. \\
		\hline
		NFR-PRF-002 & Clasificarea intenției (Choice Maker) \reqshall să se finalizeze în p95 < 500ms. \\
		\hline
		NFR-PRF-003 & Evaluarea parolei \reqshall să se finalizeze în p95 < 2s. \\
		\hline
		NFR-PRF-004 & Verificarea primalității pentru numere < 64 biți \reqshall să fie < 100ms. \\
		\hline
		NFR-PRF-005 & Operațiile criptografice standard \reqshall să se finalizeze în < 1s. \\
		\hline
		NFR-PRF-006 & Generarea RAG \reqshall să returneze răspuns în p95 < 10s (dependent de LLM). \\
		\hline
		NFR-PRF-007 & Sistemul \reqshall să suporte minim 5 de cereri concurente. \\
		\hline
		NFR-PRF-008 & Sistemul \reqshould să suporte minim 5 utilizatori concurenți activi. \\
		\hline
		NFR-PRF-009 & Cache-ul \reqshall să reducă latența pentru cereri repetitive cu minim 80\%. \\
		\hline
		NFR-PRF-010 & Operațiile heavy (factorizare, RAG extins) \reqshall să fie async cu polling. \\
		\hline
	\end{longtable}

	\subsection{Scalabilitate}

	\subsubsection{Horizontal Scaling}

	\begin{longtable}{|p{2cm}|p{12cm}|}
		\hline
		\rowcolor{tableheader}
		\textbf{ID} & \textbf{Cerință} \\
		\hline
		\endfirsthead
		\hline
		\rowcolor{tableheader}
		\textbf{ID} & \textbf{Cerință} \\
		\hline
		\endhead
		\hline
		\endfoot

		NFR-SCL-001 & Arhitectura \reqshall să permită scalare orizontală pentru toți agenții. \\
		\hline
		NFR-SCL-002 & Sistemul \reqshall să funcționeze corect cu minim 2 replici per agent critic. \\
		\hline
		NFR-SCL-003 & Baza de date \reqshall să suporte connection pooling eficient. \\
		\hline
		NFR-SCL-004 & Sistemul \reqshould să implementeze auto-scaling pe bază de load în K8s. \\
		\hline
		NFR-SCL-005 & Sistemul \reqshall să gestioneze backpressure la cereri excesive. \\
		\hline
	\end{longtable}

	\subsection{Fiabilitate}

	\subsubsection{Availability și Recovery}

	\begin{longtable}{|p{2cm}|p{12cm}|}
		\hline
		\rowcolor{tableheader}
		\textbf{ID} & \textbf{Cerință} \\
		\hline
		\endfirsthead
		\hline
		\rowcolor{tableheader}
		\textbf{ID} & \textbf{Cerință} \\
		\hline
		\endhead
		\hline
		\endfoot

		NFR-REL-001 & Disponibilitatea target pentru orchestrator și backend: \(\geq\) 99.5\%. \\
		\hline
		NFR-REL-002 & Disponibilitatea target pentru agenți individuali: \(\geq\) 99\%. \\
		\hline
		NFR-REL-003 & Sistemul \reqshall să implementeze retry cu exponential backoff pentru dependențe externe. \\
		\hline
		NFR-REL-004 & Sistemul \reqshall să implementeze circuit breaker cu threshold configurabil. \\
		\hline
		NFR-REL-005 & Sistemul \reqshall să funcționeze în mod degradat când agenți non-critici sunt indisponibili. \\
		\hline
		NFR-REL-006 & MTBF target pentru servicii critice: \(\geq\) 720 ore. \\
		\hline
		NFR-REL-007 & MTTR target: \(\leq\) 30 minute. \\
		\hline
		NFR-REL-008 & Backup-urile bazelor de date \reqshall să fie automate și testate periodic. \\
		\hline
		NFR-REL-009 & RTO (Recovery Time Objective): \(\leq\) 4 ore. \\
		\hline
		NFR-REL-010 & RPO (Recovery Point Objective): \(\leq\) 1 oră. \\
		\hline
	\end{longtable}

	\subsection{Observabilitate}

	\subsubsection{Monitoring și Logging}

	\begin{longtable}{|p{2cm}|p{12cm}|}
		\hline
		\rowcolor{tableheader}
		\textbf{ID} & \textbf{Cerință} \\
		\hline
		\endfirsthead
		\hline
		\rowcolor{tableheader}
		\textbf{ID} & \textbf{Cerință} \\
		\hline
		\endhead
		\hline
		\endfoot

		NFR-OBS-001 & Toate serviciile \reqshall să expună metrici Prometheus pe /metrics. \\
		\hline
		NFR-OBS-002 & Sistemul \reqshall să colecteze metrici RED (Rate, Errors, Duration). \\
		\hline
		NFR-OBS-003 & Sistemul \reqshall să colecteze metrici USE (Utilization, Saturation, Errors). \\
		\hline
		NFR-OBS-004 & Toate serviciile \reqshall să emită loguri structurate (JSON). \\
		\hline
		NFR-OBS-005 & Log-urile \reqshall să includă: timestamp, level, service, trace\_id, message. \\
		\hline
		NFR-OBS-006 & Sistemul \reqshould să implementeze distributed tracing (OpenTelemetry). \\
		\hline
		NFR-OBS-007 & Sistemul \reqshall să ofere dashboards Grafana pentru monitorizare. \\
		\hline
		NFR-OBS-008 & Sistemul \reqshall să configureze alerting pentru metrici critice. \\
		\hline
		NFR-OBS-009 & Alertele critice \reqshall să fie notificate în < 5 minute de la incident. \\
		\hline
		NFR-OBS-010 & Sistemul \reqshould să implementeze anomaly detection pentru pattern-uri neobișnuite. \\
		\hline
	\end{longtable}

	% ========== CERINȚE DE SECURITATE ==========
	\newpage
	\section{Cerințe de Securitate}

	\subsection{Transport Security}

	\subsubsection{TLS și mTLS}

	\begin{longtable}{|p{2cm}|p{12cm}|}
		\hline
		\rowcolor{tableheader}
		\textbf{ID} & \textbf{Cerință} \\
		\hline
		\endfirsthead
		\hline
		\rowcolor{tableheader}
		\textbf{ID} & \textbf{Cerință} \\
		\hline
		\endhead
		\hline
		\endfoot

		SR-TLS-001 & Toate comunicațiile externe \reqshall să utilizeze TLS 1.2+. \\
		\hline
		SR-TLS-002 & Comunicațiile inter-servicii în producție \reqshall să utilizeze mTLS. \\
		\hline
		SR-TLS-003 & Certificatele \reqshall să aibă minimum 2048-bit RSA sau ECDSA P-256. \\
		\hline
		SR-TLS-004 & Sistemul \reqshall să implementeze certificate rotation automată. \\
		\hline
		SR-TLS-005 & Sistemul \reqshall să forțeze HSTS cu max-age \(\geq\) 1 an. \\
		\hline
	\end{longtable}

	\subsection{Data Security}

	\subsubsection{Protecția Datelor}

	\begin{longtable}{|p{2cm}|p{12cm}|}
		\hline
		\rowcolor{tableheader}
		\textbf{ID} & \textbf{Cerință} \\
		\hline
		\endfirsthead
		\hline
		\rowcolor{tableheader}
		\textbf{ID} & \textbf{Cerință} \\
		\hline
		\endhead
		\hline
		\endfoot

		SR-DAT-001 & Datele sensibile at-rest \reqshall să fie criptate (AES-256-GCM). \\
		\hline
		SR-DAT-002 & Parolele \reqshall să fie hashuite cu bcrypt/Argon2 (cost \(\geq\) 12). \\
		\hline
		SR-DAT-003 & API keys \reqshall să fie stocate hashuite, afișate o singură dată. \\
		\hline
		SR-DAT-004 & Secretele NU \reqshall să fie stocate în cod sau imagini container. \\
		\hline
		SR-DAT-005 & Sistemul \reqshall să utilizeze secrets management (Vault/K8s Secrets). \\
		\hline
		SR-DAT-006 & Log-urile NU \reqshall să conțină date sensibile în clar. \\
		\hline
		SR-DAT-007 & Baza de date \reqshall să fie accesibilă doar din rețeaua internă. \\
		\hline
	\end{longtable}

	\subsection{Input Validation și Injection Prevention}

	\subsubsection{Validare și Sanitizare}

	\begin{longtable}{|p{2cm}|p{12cm}|}
		\hline
		\rowcolor{tableheader}
		\textbf{ID} & \textbf{Cerință} \\
		\hline
		\endfirsthead
		\hline
		\rowcolor{tableheader}
		\textbf{ID} & \textbf{Cerință} \\
		\hline
		\endhead
		\hline
		\endfoot

		SR-INJ-001 & Sistemul \reqshall să prevină SQL Injection prin parametrizare. \\
		\hline
		SR-INJ-002 & Sistemul \reqshall să prevină Command Injection prin validare strictă. \\
		\hline
		SR-INJ-003 & Sistemul \reqshall să prevină XSS prin sanitizare input și output encoding. \\
		\hline
		SR-INJ-004 & Sistemul \reqshall să prevină CSRF prin token-uri per sesiune. \\
		\hline
		SR-INJ-005 & Sistemul \reqshall să prevină Path Traversal cu validare și sandboxing. \\
		\hline
		SR-INJ-006 & Sistemul \reqshall să implementeze allowlist pentru algoritmi și operațiuni. \\
		\hline
		SR-INJ-007 & Sistemul \reqshall să valideze toate inputurile server-side. \\
		\hline
		SR-INJ-008 & Sistemul \reqshall să implementeze request size limits (default: 1MB). \\
		\hline
	\end{longtable}

	\subsection{Access Control}

	\subsubsection{Autorizare și Rate Limiting}

	\begin{longtable}{|p{2cm}|p{12cm}|}
		\hline
		\rowcolor{tableheader}
		\textbf{ID} & \textbf{Cerință} \\
		\hline
		\endfirsthead
		\hline
		\rowcolor{tableheader}
		\textbf{ID} & \textbf{Cerință} \\
		\hline
		\endhead
		\hline
		\endfoot

		SR-ACC-001 & Sistemul \reqshall să implementeze principiul privilegiilor minime. \\
		\hline
		SR-ACC-002 & Sistemul \reqshall să verifice autorizarea pentru fiecare cerere. \\
		\hline
		SR-ACC-003 & Sistemul \reqshall să implementeze rate limiting per IP și per user. \\
		\hline
		SR-ACC-004 & Sistemul \reqshall să blocheze conturile după 5 încercări eșuate (30 min). \\
		\hline
		SR-ACC-005 & Sistemul \reqshould să implementeze IP reputation și blacklisting. \\
		\hline
		SR-ACC-006 & Sistemul \reqshould să detecteze și să blocheze brute force attacks. \\
		\hline
	\end{longtable}

	\subsection{Security Headers}

	\subsubsection{HTTP Security}

	\begin{longtable}{|p{2cm}|p{12cm}|}
		\hline
		\rowcolor{tableheader}
		\textbf{ID} & \textbf{Cerință} \\
		\hline
		\endfirsthead
		\hline
		\rowcolor{tableheader}
		\textbf{ID} & \textbf{Cerință} \\
		\hline
		\endhead
		\hline
		\endfoot

		SR-HDR-001 & Sistemul \reqshall să seteze Content-Security-Policy restrictiv. \\
		\hline
		SR-HDR-002 & Sistemul \reqshall să seteze X-Frame-Options: DENY. \\
		\hline
		SR-HDR-003 & Sistemul \reqshall să seteze X-Content-Type-Options: nosniff. \\
		\hline
		SR-HDR-004 & Sistemul \reqshall să seteze Referrer-Policy: strict-origin-when-cross-origin. \\
		\hline
		SR-HDR-005 & Sistemul \reqshall să configureze CORS restrictiv (nu wildcard în producție). \\
		\hline
	\end{longtable}

	\subsection{Audit și Incident Response}

	\subsubsection{Logging și Forensics}

	\begin{longtable}{|p{2cm}|p{12cm}|}
		\hline
		\rowcolor{tableheader}
		\textbf{ID} & \textbf{Cerință} \\
		\hline
		\endfirsthead
		\hline
		\rowcolor{tableheader}
		\textbf{ID} & \textbf{Cerință} \\
		\hline
		\endhead
		\hline
		\endfoot

		SR-AUD-001 & Sistemul \reqshall să logheze toate accesele la resurse sensibile. \\
		\hline
		SR-AUD-002 & Sistemul \reqshall să logheze toate operațiunile administrative. \\
		\hline
		SR-AUD-003 & Sistemul \reqshall să păstreze audit logs imutabile pentru investigații. \\
		\hline
		SR-AUD-004 & Sistemul \reqshould să alerteze la comportament suspect (anomalii). \\
		\hline
		SR-AUD-005 & Sistemul \reqshall să permită investigație și forensics post-incident. \\
		\hline
		SR-AUD-006 & Sistemul \reqshould să ofere export pentru SIEM integration. \\
		\hline
	\end{longtable}

	% ========== CERINȚE AI/ML ==========
	\newpage
	\section{Cerințe AI/ML}

	\subsection{Model Specification}

	\subsubsection{Modele ML Utilizate}

	\begin{longtable}{|p{2cm}|p{12cm}|}
		\hline
		\rowcolor{tableheader}
		\textbf{ID} & \textbf{Cerință} \\
		\hline
		\endfirsthead
		\hline
		\rowcolor{tableheader}
		\textbf{ID} & \textbf{Cerință} \\
		\hline
		\endhead
		\hline
		\endfoot

		ML-MOD-001 & PassGPT \reqshall să utilizeze model pre-antrenat (javirandor/passgpt-10characters). \\
		\hline
		ML-MOD-002 & SecureBERT \reqshall să utilizeze versiunea 2.0 pentru clasificare. \\
		\hline
		ML-MOD-003 & Embedding model pentru RAG \reqshall să fie BAAI/bge-small-en-v1.5. \\
		\hline
		ML-MOD-004 & Reranker \reqshall să fie BAAI/bge-reranker-base (ONNX). \\
		\hline
		ML-MOD-005 & Toate modelele \reqshall să aibă checksum verificat la încărcare. \\
		\hline
		ML-MOD-006 & Modelele \reqshall să fie versionate și etichetate în registry. \\
		\hline
	\end{longtable}

	\subsection{Data Management}

	\subsubsection{Gestiunea Datelor ML}

	\begin{longtable}{|p{2cm}|p{12cm}|}
		\hline
		\rowcolor{tableheader}
		\textbf{ID} & \textbf{Cerință} \\
		\hline
		\endfirsthead
		\hline
		\rowcolor{tableheader}
		\textbf{ID} & \textbf{Cerință} \\
		\hline
		\endhead
		\hline
		\endfoot

		ML-DAT-001 & Documentele ingestate \reqshall să fie clasificate și etichetate. \\
		\hline
		ML-DAT-002 & Sistemul \reqshall să păstreze metadata pentru fiecare document. \\
		\hline
		ML-DAT-003 & Sistemul \reqshould să permită actualizarea incrementală a vectorilor. \\
		\hline
		ML-DAT-004 & Sistemul \reqshall să permită ștergerea selectivă din vector store. \\
		\hline
		ML-DAT-005 & Dataset-urile de antrenament \reqshall să fie documentate și versionate. \\
		\hline
	\end{longtable}

	\subsection{Guardrails și Safety}

	\subsubsection{Protecție și Siguranță ML}

	\begin{longtable}{|p{2cm}|p{12cm}|}
		\hline
		\rowcolor{tableheader}
		\textbf{ID} & \textbf{Cerință} \\
		\hline
		\endfirsthead
		\hline
		\rowcolor{tableheader}
		\textbf{ID} & \textbf{Cerință} \\
		\hline
		\endhead
		\hline
		\endfoot

		ML-GRD-001 & Sistemul \reqshall să valideze inputul înainte de procesare ML. \\
		\hline
		ML-GRD-002 & Sistemul \reqshall să limiteze lungimea inputului acceptat (context window). \\
		\hline
		ML-GRD-003 & Sistemul \reqshall să filtreze output-urile pentru conținut harmful. \\
		\hline
		ML-GRD-004 & Sistemul \reqshall să implementeze limită de acțiuni per sesiune. \\
		\hline
		ML-GRD-005 & Sistemul \reqshould să detecteze și să blocheze prompt injection attempts. \\
		\hline
		ML-GRD-006 & Sistemul NU \reqshall să expună informații sensibile prin model outputs. \\
		\hline
	\end{longtable}

	\subsection{Model Lifecycle (MLOps)}

	\subsubsection{Management și Deployment Modele}

	\begin{longtable}{|p{2cm}|p{12cm}|}
		\hline
		\rowcolor{tableheader}
		\textbf{ID} & \textbf{Cerință} \\
		\hline
		\endfirsthead
		\hline
		\rowcolor{tableheader}
		\textbf{ID} & \textbf{Cerință} \\
		\hline
		\endhead
		\hline
		\endfoot

		ML-OPS-001 & Sistemul \reqshall să suporte blue-green deployment pentru modele. \\
		\hline
		ML-OPS-002 & Sistemul \reqshall să monitorizeze drift-ul modelelor. \\
		\hline
		ML-OPS-003 & Sistemul \reqshould să implementeze A/B testing pentru modele noi. \\
		\hline
		ML-OPS-004 & Sistemul \reqshall să permită rollback rapid la versiunea anterioară. \\
		\hline
		ML-OPS-005 & Sistemul \reqshall să păstreze metrici de performanță per versiune model. \\
		\hline
	\end{longtable}

	\subsection{Ethics și Transparency}

	\subsubsection{Transparență AI}

	\begin{longtable}{|p{2cm}|p{12cm}|}
		\hline
		\rowcolor{tableheader}
		\textbf{ID} & \textbf{Cerință} \\
		\hline
		\endfirsthead
		\hline
		\rowcolor{tableheader}
		\textbf{ID} & \textbf{Cerință} \\
		\hline
		\endhead
		\hline
		\endfoot

		ML-ETH-001 & Sistemul \reqshall să informeze utilizatorii că răspunsurile sunt generate de AI. \\
		\hline
		ML-ETH-002 & Sistemul \reqshall să ofere confidence scores pentru predicții. \\
		\hline
		ML-ETH-003 & Sistemul \reqshould să documenteze limitările cunoscute ale modelelor. \\
		\hline
		ML-ETH-004 & Sistemul NU \reqshall să pretindă certitudine pentru rezultate probabilistice. \\
		\hline
	\end{longtable}

	% ========== CAZURI DE UTILIZARE ==========
	\newpage
	\section{Cazuri de Utilizare}

	\subsection{Anonymous}

	\begin{longtable}{|p{2cm}|p{12cm}|}
		\hline
		\rowcolor{tableheader}
		\textbf{ID} & \textbf{Cerință} \\
		\hline
		\endfirsthead
		\hline
		\rowcolor{tableheader}
		\textbf{ID} & \textbf{Cerință} \\
		\hline
		\endhead
		\hline
		\endfoot

		UC-ANON-001 & Utilizatorul Anonymous \reqshall să poată accesa prezentarea publică a platformei și lista de unelte/agenți. \\
		\hline
		UC-ANON-002 & Utilizatorul Anonymous \reqshall să poată crea un cont nou. \\
		\hline
		UC-ANON-003 & Utilizatorul Anonymous \reqshall să poată iniția autentificarea și recuperarea parolei. \\
		\hline
		UC-ANON-004 & Utilizatorul Anonymous \reqshall să poată rula o demonstrație limitată pentru Password Checker. \\
		\hline
		UC-ANON-005 & Utilizatorul Anonymous \reqshall să poată rula o demonstrație limitată pentru Prime Checker (numere mici). \\
		\hline
		UC-ANON-006 & Utilizatorul Anonymous \reqshall să poată rula o demonstrație limitată de detectare criptosisteme cu scor de încredere. \\
		\hline
		UC-ANON-007 & Utilizatorul Anonymous \reqshall să poată consulta statusul serviciilor și documentația publică API/CLI. \\
		\hline
	\end{longtable}

	\subsection{User}

	\begin{longtable}{|p{2cm}|p{12cm}|}
		\hline
		\rowcolor{tableheader}
		\textbf{ID} & \textbf{Cerință} \\
		\hline
		\endfirsthead
		\hline
		\rowcolor{tableheader}
		\textbf{ID} & \textbf{Cerință} \\
		\hline
		\endhead
		\hline
		\endfoot

		UC-USER-001 & Utilizatorul User \reqshall să poată gestiona sesiunea (login, refresh token, logout). \\
		\hline
		UC-USER-002 & Utilizatorul User \reqshall să poată trimite cereri către Orchestrator pentru rutare multi-agent și răspuns agregat. \\
		\hline
		UC-USER-003 & Utilizatorul User \reqshall să poată audita parole prin agregatorul ML (PassGPT, zxcvbn, HIBP). \\
		\hline
		UC-USER-004 & Utilizatorul User \reqshall să poată iniția job-uri Hash Breaker și să urmărească statusul lor asincron. \\
		\hline
		UC-USER-005 & Utilizatorul User \reqshall să poată verifica primalitatea și factorizarea numerelor și să acceseze istoricul cererilor. \\
		\hline
		UC-USER-006 & Utilizatorul User \reqshall să poată executa operații criptografice (encode/hash/AES/RSA/PQC) prin Command Executor. \\
		\hline
		UC-USER-007 & Utilizatorul User \reqshall să poată utiliza modul educațional (Theory Specialist RAG + CTF Tool) pentru întrebări și exerciții. \\
		\hline
	\end{longtable}

	\subsection{Admin}

	\begin{longtable}{|p{2cm}|p{12cm}|}
		\hline
		\rowcolor{tableheader}
		\textbf{ID} & \textbf{Cerință} \\
		\hline
		\endfirsthead
		\hline
		\rowcolor{tableheader}
		\textbf{ID} & \textbf{Cerință} \\
		\hline
		\endhead
		\hline
		\endfoot

		UC-ADMIN-001 & Utilizatorul Admin \reqshall să poată administra utilizatorii (creare, blocare, resetare). \\
		\hline
		UC-ADMIN-002 & Utilizatorul Admin \reqshall să poată configura roluri și permisiuni RBAC. \\
		\hline
		UC-ADMIN-003 & Utilizatorul Admin \reqshall să poată genera și revoca chei API pentru integrări externe. \\
		\hline
		UC-ADMIN-004 & Utilizatorul Admin \reqshall să poată configura politici de acces și rate limiting. \\
		\hline
		UC-ADMIN-005 & Utilizatorul Admin \reqshall să poată administra corpusul RAG (ingestie, reindexare, ștergere). \\
		\hline
		UC-ADMIN-006 & Utilizatorul Admin \reqshall să poată monitoriza serviciile, metricile și alertele operaționale. \\
		\hline
		UC-ADMIN-007 & Utilizatorul Admin \reqshall să poată consulta și exporta audit logs pentru investigații. \\
		\hline
	\end{longtable}

	\subsection{Sistem Extern (API Client)}

	\begin{longtable}{|p{2cm}|p{12cm}|}
		\hline
		\rowcolor{tableheader}
		\textbf{ID} & \textbf{Cerință} \\
		\hline
		\endfirsthead
		\hline
		\rowcolor{tableheader}
		\textbf{ID} & \textbf{Cerință} \\
		\hline
		\endhead
		\hline
		\endfoot

		UC-EXT-001 & Sistemul extern \reqshall să poată autentifica cereri folosind API key sau token JWT. \\
		\hline
		UC-EXT-002 & Sistemul extern \reqshall să poată apela endpoint-ul /v1/orchestrate și să primească răspuns agregat. \\
		\hline
		UC-EXT-003 & Sistemul extern \reqshall să poată apela direct agenții (password/prime/cryptosystem/command executor). \\
		\hline
		UC-EXT-004 & Sistemul extern \reqshall să poată iniția job-uri Hash Breaker și să interogheze statusul lor. \\
		\hline
		UC-EXT-005 & Sistemul extern \reqshall să poată interoga Theory Specialist (RAG) și să primească sursele folosite. \\
		\hline
		UC-EXT-006 & Sistemul extern \reqshall să poată consulta endpoint-uri de health/status pentru monitorizare. \\
		\hline
		UC-EXT-007 & Sistemul extern \reqshall să primească coduri de eroare standard și headere de rate limiting pentru retry/backoff. \\
		\hline
	\end{longtable}

	% ========== CERINȚE INFRASTRUCTURĂ ==========
	\newpage
	\section{Cerințe de Infrastructură și DevOps}

	\subsection{Containerizare și Orchestrare}

	\subsubsection{Kubernetes}

	\begin{longtable}{|p{2cm}|p{12cm}|}
		\hline
		\rowcolor{tableheader}
		\textbf{ID} & \textbf{Cerință} \\
		\hline
		\endfirsthead
		\hline
		\rowcolor{tableheader}
		\textbf{ID} & \textbf{Cerință} \\
		\hline
		\endhead
		\hline
		\endfoot

		INF-K8S-001 & Sistemul \reqshall să ruleze în Kubernetes cu namespace segregation. \\
		\hline
		INF-K8S-002 & Sistemul \reqshall să definească resource limits pentru toate container-ele. \\
		\hline
		INF-K8S-003 & Sistemul \reqshall să implementeze network policies pentru izolare. \\
		\hline
		INF-K8S-004 & Sistemul \reqshall să utilizeze non-root containers. \\
		\hline
		INF-K8S-005 & Sistemul \reqshall să implementeze pod security standards. \\
		\hline
		INF-K8S-006 & Sistemul \reqshould să utilizeze service mesh (Istio/Linkerd). \\
		\hline
		INF-K8S-007 & Sistemul \reqshall să implementeze health checks (liveness/readiness). \\
		\hline
		INF-K8S-008 & Sistemul \reqshould să suporte horizontal pod autoscaling. \\
		\hline
	\end{longtable}

	\subsection{CI/CD Pipeline}

	\subsubsection{Automatizare Build și Deploy}

	\begin{longtable}{|p{2cm}|p{12cm}|}
		\hline
		\rowcolor{tableheader}
		\textbf{ID} & \textbf{Cerință} \\
		\hline
		\endfirsthead
		\hline
		\rowcolor{tableheader}
		\textbf{ID} & \textbf{Cerință} \\
		\hline
		\endhead
		\hline
		\endfoot

		INF-CIC-001 & Pipeline \reqshall să execute build automat la fiecare commit. \\
		\hline
		INF-CIC-002 & Pipeline \reqshall să execute unit tests cu coverage \(\geq\) 70\%. \\
		\hline
		INF-CIC-003 & Pipeline \reqshall să execute static analysis (linters). \\
		\hline
		INF-CIC-004 & Pipeline \reqshall să execute security scanning (Trivy, Snyk). \\
		\hline
		INF-CIC-005 & Pipeline \reqshall să execute integration tests. \\
		\hline
		INF-CIC-006 & Pipeline \reqshould să execute SAST și DAST. \\
		\hline
		INF-CIC-007 & Pipeline \reqshall să genereze și să publice imagini cu tag semantic. \\
		\hline
		INF-CIC-008 & Pipeline \reqshall să implementeze deployment automat în staging. \\
		\hline
		INF-CIC-009 & Pipeline \reqshould să suporte canary deployments în producție. \\
		\hline
		INF-CIC-010 & Pipeline \reqshall să permită rollback rapid (< 5 minute). \\
		\hline
	\end{longtable}

	\subsection{Deployment și Portability}

	\subsubsection{Portabilitate și Instalare}

	\begin{longtable}{|p{2cm}|p{12cm}|}
		\hline
		\rowcolor{tableheader}
		\textbf{ID} & \textbf{Cerință} \\
		\hline
		\endfirsthead
		\hline
		\rowcolor{tableheader}
		\textbf{ID} & \textbf{Cerință} \\
		\hline
		\endhead
		\hline
		\endfoot

		INF-DEP-001 & Sistemul \reqshall să suporte deployment on-premise. \\
		\hline
		INF-DEP-002 & Sistemul \reqshould să suporte deployment în cloud (AWS/GCP/Azure). \\
		\hline
		INF-DEP-003 & Sistemul \reqshall să funcționeze pe Linux (Ubuntu 22.04+, Debian 12+). \\
		\hline
		INF-DEP-004 & Sistemul \reqshould să suporte air-gapped deployment. \\
		\hline
		INF-DEP-005 & Configurația \reqshall să fie externalizată prin env vars/ConfigMaps. \\
		\hline
		INF-DEP-006 & Sistemul \reqshall să ofere documentație completă pentru deployment. \\
		\hline
	\end{longtable}

	\subsection{Operational Readiness}

	\subsubsection{Producție}

	\begin{longtable}{|p{2cm}|p{12cm}|}
		\hline
		\rowcolor{tableheader}
		\textbf{ID} & \textbf{Cerință} \\
		\hline
		\endfirsthead
		\hline
		\rowcolor{tableheader}
		\textbf{ID} & \textbf{Cerință} \\
		\hline
		\endhead
		\hline
		\endfoot

		INF-OPS-001 & Sistemul \reqshall să aibă runbook-uri pentru incident response și operațiuni critice. \\
		\hline
		INF-OPS-002 & Procedurile de backup și restore \reqshall să fie documentate și testate periodic. \\
		\hline
		INF-OPS-003 & Sistemul \reqshall să aibă plan de disaster recovery cu RTO/RPO validate. \\
		\hline
		INF-OPS-004 & Audit log-urile \reqshall să fie protejate împotriva modificării și accesate doar cu roluri dedicate. \\
		\hline
		INF-OPS-005 & Release-urile \reqshall să treacă prin quality gates (teste, scanări, verificări de securitate). \\
		\hline
		INF-OPS-006 & Sistemul \reqshall să efectueze audit de securitate periodic (cel puțin anual sau per release major). \\
		\hline
		INF-OPS-007 & Alertele critice \reqshall să fie rutate către un canal de on-call. \\
		\hline
	\end{longtable}

	% ========== CERINȚE DE CONFORMITATE ==========
	\newpage
	\section{Cerințe de Conformitate}

	\subsection{Security Standards}

	\subsubsection{Conformitate Securitate}

	\begin{longtable}{|p{2cm}|p{12cm}|}
		\hline
		\rowcolor{tableheader}
		\textbf{ID} & \textbf{Cerință} \\
		\hline
		\endfirsthead
		\hline
		\rowcolor{tableheader}
		\textbf{ID} & \textbf{Cerință} \\
		\hline
		\endhead
		\hline
		\endfoot

		CMP-SEC-001 & Sistemul \reqshall să respecte OWASP Top 10 (2021). \\
		\hline
		CMP-SEC-002 & Sistemul \reqshould să respecte CIS Benchmarks pentru containerizare. \\
		\hline
		CMP-SEC-003 & Sistemul \reqshould să respecte NIST Cybersecurity Framework. \\
		\hline
		CMP-SEC-004 & Operațiunile criptografice \reqshould să respecte NIST SP 800-57. \\
		\hline
		CMP-SEC-005 & Post-quantum crypto \reqshould să respecte NIST PQC standards. \\
		\hline
	\end{longtable}

	\subsection{Data Protection}

	\subsubsection{GDPR Compliance}

	\begin{longtable}{|p{2cm}|p{12cm}|}
		\hline
		\rowcolor{tableheader}
		\textbf{ID} & \textbf{Cerință} \\
		\hline
		\endfirsthead
		\hline
		\rowcolor{tableheader}
		\textbf{ID} & \textbf{Cerință} \\
		\hline
		\endhead
		\hline
		\endfoot

		CMP-GDP-001 & Sistemul \reqshall să permită exercitarea dreptului la ștergere (Art. 17 GDPR). \\
		\hline
		CMP-GDP-002 & Sistemul \reqshall să permită exportul datelor personale (Art. 20 GDPR). \\
		\hline
		CMP-GDP-003 & Sistemul \reqshall să documenteze fluxurile de date personale. \\
		\hline
		CMP-GDP-004 & Sistemul \reqshall să minimizeze colectarea datelor (Art. 5 GDPR). \\
		\hline
		CMP-GDP-005 & Sistemul \reqshould să implementeze pseudonimizare unde posibil. \\
		\hline
	\end{longtable}

	% ========== ANEXE ==========
	\newpage
	\section{Anexă}

	\subsection{Diagrame UML}

	\subsubsection{Arhitectură generală}
	Arhitectura platformei software de unelte AI în criptografie este organizată pe niveluri:

	\begin{enumerate}
		\item \textbf{Nivel Frontend} - React Web, CLI Client, API Consumers
		\item \textbf{Nivel API Gateway} - Go Backend (Auth, Rate Limiting, Routing)
		\item \textbf{Nivel Orchestrare} - Orchestrator (Intent Routing, LLM)
		\item \textbf{Nivel Agenți} - agenți specializați
		\item \textbf{Nivel Date} - PostgreSQL, Redis, ChromaDB, BoltDB
	\end{enumerate}
	\begin{figure}[H]
		\centering
		\includesvg[width=\textwidth]{diagrama_arhitectura}
		\caption{Arhitectura generală a platformei}
	\end{figure}

	\subsubsection{Cazuri de utilizare - Anonymous}
	\begin{figure}[H]
		\centering
		\includesvg[width=0.9\textwidth,height=0.40\textheight,keepaspectratio]{diagrama_use_case_anonymous}
		\caption{Cazuri de utilizare pentru Anonymous}
	\end{figure}

	\subsubsection{Cazuri de utilizare - User}
	\begin{figure}[H]
		\centering
		\includesvg[width=0.9\textwidth,height=0.40\textheight,keepaspectratio]{diagrama_use_case_user}
		\caption{Cazuri de utilizare pentru User}
	\end{figure}

	\subsubsection{Cazuri de utilizare - Admin}
	\begin{figure}[H]
		\centering
		\includesvg[width=0.9\textwidth,height=0.40\textheight,keepaspectratio]{diagrama_use_case_admin}
		\caption{Cazuri de utilizare pentru Admin}
	\end{figure}

	\subsubsection{Cazuri de utilizare - Sistem extern}
	\begin{figure}[H]
		\centering
		\includesvg[width=0.9\textwidth,height=0.40\textheight,keepaspectratio]{diagrama_use_case_external_system}
		\caption{Cazuri de utilizare pentru Sistem Extern (API Client)}
	\end{figure}

	\subsubsection{Flux secvențial cerere-răspuns}
	\begin{figure}[H]
		\centering
		\includesvg[width=\textwidth]{diagrama_sequence_request_flow}
		\caption{Flux secvențial pentru o cerere standard}
	\end{figure}

	\subsubsection{Flux asincron pentru operații heavy}
	\begin{figure}[H]
		\centering
		\includesvg[width=0.9\textwidth]{diagrama_activity_async_jobs}
		\caption{Flux asincron pentru operații heavy}
	\end{figure}

	\subsubsection{Ciclul de viață al job-urilor}
	\begin{figure}[H]
		\centering
		\includesvg[width=0.9\textwidth]{diagrama_state_job_lifecycle}
		\caption{Lifecycle job-uri (queued, running, completed/failed)}
	\end{figure}

	\subsubsection{Agent Verificare Parole}
	\begin{figure}[H]
		\centering
		\includesvg[width=\textwidth]{diagrama_password_checker}
		\caption{Diagramă componentă pentru Password Checker}
	\end{figure}

	\subsubsection{Agent Verificare Primalitate}
	\begin{figure}[H]
		\centering
		\includesvg[width=\textwidth]{diagrama_prime_checker}
		\caption{Diagramă componentă pentru Prime Checker}
	\end{figure}

	\subsubsection{Agent Specialist Teorie (RAG)}
	\begin{figure}[H]
		\centering
		\includesvg[width=\textwidth]{diagrama_theory_specialist}
		\caption{Diagramă componentă pentru Theory Specialist}
	\end{figure}

	\subsubsection{Agent Executor Comenzi}
	\begin{figure}[H]
		\centering
		\includesvg[width=\textwidth]{diagrama_command_executor}
		\caption{Diagramă componentă pentru Command Executor}
	\end{figure}

	\subsubsection{Agent Choice Maker (clasificare)}
	\begin{figure}[H]
		\centering
		\includesvg[width=0.85\textwidth]{diagrama_choice_maker_make_choice}
		\caption{Diagramă componentă pentru Choice Maker (inferență)}
	\end{figure}

	\subsubsection{Agent Choice Maker (generare întrebări)}
	\begin{figure}[H]
		\centering
		\includesvg[width=\textwidth]{diagrama_choice_maker_generate_questions}
		\caption{Diagramă componentă pentru Choice Maker (training)}
	\end{figure}

	\subsubsection{Agent Detectare Criptosisteme}
	\begin{figure}[H]
		\centering
		\includesvg[width=\textwidth]{diagrama_cryptosystem_detection}
		\caption{Diagramă componentă pentru Cryptosystem Detection}
	\end{figure}

	% ========== SFÂRȘIT DOCUMENT ==========
\end{document}
